\documentclass[english,man]{apa6}

\usepackage{amssymb,amsmath}
\usepackage{ifxetex,ifluatex}
\usepackage{fixltx2e} % provides \textsubscript
\ifnum 0\ifxetex 1\fi\ifluatex 1\fi=0 % if pdftex
  \usepackage[T1]{fontenc}
  \usepackage[utf8]{inputenc}
\else % if luatex or xelatex
  \ifxetex
    \usepackage{mathspec}
    \usepackage{xltxtra,xunicode}
  \else
    \usepackage{fontspec}
  \fi
  \defaultfontfeatures{Mapping=tex-text,Scale=MatchLowercase}
  \newcommand{\euro}{€}
\fi
% use upquote if available, for straight quotes in verbatim environments
\IfFileExists{upquote.sty}{\usepackage{upquote}}{}
% use microtype if available
\IfFileExists{microtype.sty}{\usepackage{microtype}}{}

% Table formatting
\usepackage{longtable, booktabs}
\usepackage{lscape}
% \usepackage[counterclockwise]{rotating}   % Landscape page setup for large tables
\usepackage{multirow}		% Table styling
\usepackage{tabularx}		% Control Column width
\usepackage[flushleft]{threeparttable}	% Allows for three part tables with a specified notes section
\usepackage{threeparttablex}            % Lets threeparttable work with longtable

% Create new environments so endfloat can handle them
% \newenvironment{ltable}
%   {\begin{landscape}\begin{center}\begin{threeparttable}}
%   {\end{threeparttable}\end{center}\end{landscape}}

\newenvironment{lltable}
  {\begin{landscape}\begin{center}\begin{ThreePartTable}}
  {\end{ThreePartTable}\end{center}\end{landscape}}

  \usepackage{ifthen} % Only add declarations when endfloat package is loaded
  \ifthenelse{\equal{\string man}{\string man}}{%
   \DeclareDelayedFloatFlavor{ThreePartTable}{table} % Make endfloat play with longtable
   % \DeclareDelayedFloatFlavor{ltable}{table} % Make endfloat play with lscape
   \DeclareDelayedFloatFlavor{lltable}{table} % Make endfloat play with lscape & longtable
  }{}%



% The following enables adjusting longtable caption width to table width
% Solution found at http://golatex.de/longtable-mit-caption-so-breit-wie-die-tabelle-t15767.html
\makeatletter
\newcommand\LastLTentrywidth{1em}
\newlength\longtablewidth
\setlength{\longtablewidth}{1in}
\newcommand\getlongtablewidth{%
 \begingroup
  \ifcsname LT@\roman{LT@tables}\endcsname
  \global\longtablewidth=0pt
  \renewcommand\LT@entry[2]{\global\advance\longtablewidth by ##2\relax\gdef\LastLTentrywidth{##2}}%
  \@nameuse{LT@\roman{LT@tables}}%
  \fi
\endgroup}


\ifxetex
  \usepackage[setpagesize=false, % page size defined by xetex
              unicode=false, % unicode breaks when used with xetex
              xetex]{hyperref}
\else
  \usepackage[unicode=true]{hyperref}
\fi
\hypersetup{breaklinks=true,
            pdfauthor={},
            pdftitle={Modeling Social Attitudes with Beta Regression Hurdle Models},
            colorlinks=true,
            citecolor=blue,
            urlcolor=blue,
            linkcolor=black,
            pdfborder={0 0 0}}
\urlstyle{same}  % don't use monospace font for urls

\setlength{\parindent}{0pt}
%\setlength{\parskip}{0pt plus 0pt minus 0pt}

\setlength{\emergencystretch}{3em}  % prevent overfull lines

\ifxetex
  \usepackage{polyglossia}
  \setmainlanguage{}
\else
  \usepackage[english]{babel}
\fi

% Manuscript styling
\captionsetup{font=singlespacing,justification=justified}
\usepackage{csquotes}
\usepackage{upgreek}



\usepackage{tikz} % Variable definition to generate author note

% fix for \tightlist problem in pandoc 1.14
\providecommand{\tightlist}{%
  \setlength{\itemsep}{0pt}\setlength{\parskip}{0pt}}

% Essential manuscript parts
  \title{Modeling Social Attitudes with Beta Regression Hurdle Models}

  \shorttitle{Beta Hurdles}


  \author{Mark H. White II\textsuperscript{1}}

  \def\affdep{{""}}%
  \def\affcity{{""}}%

  \affiliation{
    \vspace{0.5cm}
          \textsuperscript{1} University of Kansas  }

  \authornote{
    \newcounter{author}
    Author note will go here.

                      Correspondence concerning this article should be addressed to Mark H. White II. E-mail: \href{mailto:markhwhiteii@gmail.com}{\nolinkurl{markhwhiteii@gmail.com}}
                }


  \abstract{Abstract will go here.}
  \keywords{beta regression, hurdle models, norms, social attitudes \\

    
  }





\usepackage{amsthm}
\newtheorem{theorem}{Theorem}
\newtheorem{lemma}{Lemma}
\theoremstyle{definition}
\newtheorem{definition}{Definition}
\newtheorem{corollary}{Corollary}
\newtheorem{proposition}{Proposition}
\theoremstyle{definition}
\newtheorem{example}{Example}
\theoremstyle{remark}
\newtheorem*{remark}{Remark}
\begin{document}

\maketitle

\setcounter{secnumdepth}{0}



\subsection{The Beta Distribution}\label{the-beta-distribution}

The beta distribution can be used to model the residuals in a
generalized linear model when the values of the dependent variable are
bounded \(0 < y < 1\) (Coxe, West, \& Aiken, 2013). The probability
density function (pdf) of the beta distribution is determined by two
parameters, \(\alpha\) and \(\beta\), that are called \enquote{shape}
parameters:

\begin{center}
$f(y;\alpha,\beta)={\Gamma(\alpha+\beta)\over\Gamma(\alpha)\Gamma(\beta)}y^{\alpha-1}(1-y)^{\beta-1}$
\end{center}

where \(\Gamma(.)\) is the gamma function. The two shape parameters pull
the mean toward zero and one, respectively. If \(\alpha\) is larger than
\(\beta\), the mean leans toward zero; if the reverse is true, the mean
leans toward one. One of the benefits of the beta distribution is that
it is flexible and can take a number of distributional shapes. ADD
FIGURE 1 HERE.

These parameters are not intuitive to applied researchers used to using
regression and analysis of variance, however. Rigby, Stasinopoulos,
Heller, and De Bastiani (2017) \enquote{reparameterized} the beta
distribution to make it easier understand in a regression framework (but
see Ferrari and Cribari-Neto (2004) for a different parameterization).
Instead of predicting \(\alpha\) and \(\beta\), they parameterize the
beta distribution with two different parameters: \(\mu\) (called the
\enquote{location} parameter) and \(\sigma\) (called the \enquote{scale}
parameter), where \(\mu={\alpha\over(\alpha+\beta)}\) and
\(\sigma=\sqrt{1\over(\alpha+\beta+1)}\). \(\mu\) is equivalent to the
mean, and \(\sigma\) is related positively to the variance. (Note that
\(\sigma\) is \emph{not} the standard deviation, even though \(\sigma\)
refers to the standard deviation in many other contexts.) The variance
is equivalent to \(\sigma^2\mu(1-\mu)\). There are two important things
to note from this equation: First, the greater the \(\sigma\), the
greater the variance; Second, the variance depends on the mean. This
means that beta regression, covered shortly, will be naturally
heteroskedastic.

But how can we model dependent variables that equal zero and/or one?

\subsection{Zero-One Inflated Beta
Distribution}\label{zero-one-inflated-beta-distribution}

Rigby et al. (2017) show that the beta distribution can be
\enquote{inflated} at zero or one---that is, zeros and ones can be
modeled. This distribution is a mixture distribution. When the dependent
variable contains zeroes and ones (i.e., \(0 \leq y \leq 1\)), the pdf
for this beta mixture, \(\text{beinf}\), is:

\begin{center}
\[
\text{beinf}(y;\mu,\sigma,\nu,\tau) =
\begin{cases}
  p_0                             & \text{if } y = 0\\
  (1 - p_0 - p_1)f(y;\mu,\sigma)  & \text{if } 0 < y < 1\\
  p_1                             & \text{if } y = 1
\end{cases}
\]
\end{center}

for \(0 \leq y \leq 1\), where \(f(y;\mu,\sigma)\) is the beta pdf with
\(\mu\) and \(\sigma\) bounded \emph{between} zero and one. The two
additional parameters, \(\nu\) and \(\tau\), are mixture parameters.
\(p_0\) is the probability that a case equals zero, \(p_1\) is the
probability that a case equals one, and \(p_2\) (i.e.,
\(1 - p_0 - p_1\)) is the probability that the case comes from the beta
distribution. In terms of these two additional parameters,
\(p_0 = {\nu \over (1 + \nu + \tau)}\) and
\(p_1 = {\tau \over (1 + \nu + \tau)}\). Rearranging these
algebraically, \(\nu\) is the odds that a case is zero to being from the
beta distribution, \(\nu = p_0 / p_2\), and \(\tau\) is the odds that a
case is a one to being from the beta distribution, \(\tau = p_1 / p_2\).

\subsection{Beta Regression Models}\label{beta-regression-models}

The goal now is to predict these four parameters,
\(\mu, \sigma, \nu, \tau\), from any number of predictor variables. All
four parameters can be predicted by an identical set of predictors, none
of the same predictors, or anywhere in between. Both \(\mu\) and
\(\sigma\) have to be between zero and one, so we can use the logistic
link function to fit predicted values in this range; both \(\nu\) and
\(\tau\) have to be greater than zero, so we can use the log link
function to fit predicted values in this range. Imagine we have one
predictor, \(x\). We could include this variable as a predictor of all
four variables with the equations:

\begin{center}
$\text{log}({\mu \over 1 - \mu}) = \beta_{10} + \beta_{11}X$\\
$\text{log}({\sigma \over 1 - \sigma}) = \beta_{20} + \beta_{21}X$\\
$\text{log}(\nu) = \beta_{30} + \beta_{31}X$\\
$\text{log}(\tau) = \beta_{40} + \beta_{41}X$
\end{center}

Or, equivalently stated:

\begin{center}
$\mu={1\over1+e^{-(\beta_{10}+\beta_{11}X)}}$\\
$\sigma={1\over1+e^{-(\beta_{20}+\beta_{21}X)}}$\\
$\nu = e^{\beta_{30} + \beta_{31}X}$\\
$\tau = e^{\beta_{40} + \beta_{41}X}$
\end{center}

where \(e^x\) is the natural exponential function. A regression model
can also be used when the dependent variable contains zeros but no ones
(e.g., \(0 \leq y < 1\)) or when it contains ones but no zeros (e.g.,
\(0 < y \leq 1\)). Let \(c\) be the value---0 or 1---that is inflated.
The pdf is:

\begin{center}
\[
\text{beinf}_c(y;\mu,\sigma,\nu) =
\begin{cases}
  p_c                             & \text{if } y = c\\
  (1 - p_c)f(y;\mu,\sigma)        & \text{if } 0 < y < 1\\
\end{cases}
\]
\end{center}

where \(\nu = p_c / (1 - p_c)\) and everything else is the same as
above. The same link functions are used as above, but the fourth
parameter, \(\tau\), is not included, as the dependent variable can only
take on values of \(c\) or from the beta distribution. Lastly, if no
zeros or ones are observed, the beta distribution alone can be used as
the pdf. This results in a beta regression where the researcher is only
predicting the location, \(\mu\), and shape, \(\sigma\). Since there is
no mixture with values of being 0 or 1, the latter two mixture
parameters, \(\nu\) and \(\tau\), are not included.

\subsection{Applications to Social and Personality
Psychology}\label{applications-to-social-and-personality-psychology}

\newpage

\section{References}\label{references}

\setlength{\parindent}{-0.5in} \setlength{\leftskip}{0.5in}

\hypertarget{refs}{}
\hypertarget{ref-coxe2013generalized}{}
Coxe, S., West, S. G., \& Aiken, L. S. (2013). Generalized linear
models. In T. D. Little (Ed.), \emph{The Oxford handbook of quantitative
methods, volume 2} (pp. 26--51). New York, NY: Oxford University Press.

\hypertarget{ref-ferrari2004beta}{}
Ferrari, S., \& Cribari-Neto, F. (2004). Beta regression for modelling
rates and proportions. \emph{Journal of Applied Statistics},
\emph{31}(7), 799--815.

\hypertarget{ref-rigby2017distributions}{}
Rigby, R., Stasinopoulos, M., Heller, G., \& De Bastiani, F. (2017).
\emph{Distributions for modelling location, scale, and shape: Using
GAMLSS in R. Unpublished draft}. Retrieved from
\url{gamlss.com/books-articles}






\end{document}
