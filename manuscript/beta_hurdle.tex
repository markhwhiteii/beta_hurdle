\documentclass[english,man]{apa6}

\usepackage{amssymb,amsmath}
\usepackage{ifxetex,ifluatex}
\usepackage{fixltx2e} % provides \textsubscript
\ifnum 0\ifxetex 1\fi\ifluatex 1\fi=0 % if pdftex
  \usepackage[T1]{fontenc}
  \usepackage[utf8]{inputenc}
\else % if luatex or xelatex
  \ifxetex
    \usepackage{mathspec}
    \usepackage{xltxtra,xunicode}
  \else
    \usepackage{fontspec}
  \fi
  \defaultfontfeatures{Mapping=tex-text,Scale=MatchLowercase}
  \newcommand{\euro}{€}
\fi
% use upquote if available, for straight quotes in verbatim environments
\IfFileExists{upquote.sty}{\usepackage{upquote}}{}
% use microtype if available
\IfFileExists{microtype.sty}{\usepackage{microtype}}{}
\usepackage{color}
\usepackage{fancyvrb}
\newcommand{\VerbBar}{|}
\newcommand{\VERB}{\Verb[commandchars=\\\{\}]}
\DefineVerbatimEnvironment{Highlighting}{Verbatim}{commandchars=\\\{\}}
% Add ',fontsize=\small' for more characters per line
\usepackage{framed}
\definecolor{shadecolor}{RGB}{248,248,248}
\newenvironment{Shaded}{\begin{snugshade}}{\end{snugshade}}
\newcommand{\KeywordTok}[1]{\textcolor[rgb]{0.13,0.29,0.53}{\textbf{#1}}}
\newcommand{\DataTypeTok}[1]{\textcolor[rgb]{0.13,0.29,0.53}{#1}}
\newcommand{\DecValTok}[1]{\textcolor[rgb]{0.00,0.00,0.81}{#1}}
\newcommand{\BaseNTok}[1]{\textcolor[rgb]{0.00,0.00,0.81}{#1}}
\newcommand{\FloatTok}[1]{\textcolor[rgb]{0.00,0.00,0.81}{#1}}
\newcommand{\ConstantTok}[1]{\textcolor[rgb]{0.00,0.00,0.00}{#1}}
\newcommand{\CharTok}[1]{\textcolor[rgb]{0.31,0.60,0.02}{#1}}
\newcommand{\SpecialCharTok}[1]{\textcolor[rgb]{0.00,0.00,0.00}{#1}}
\newcommand{\StringTok}[1]{\textcolor[rgb]{0.31,0.60,0.02}{#1}}
\newcommand{\VerbatimStringTok}[1]{\textcolor[rgb]{0.31,0.60,0.02}{#1}}
\newcommand{\SpecialStringTok}[1]{\textcolor[rgb]{0.31,0.60,0.02}{#1}}
\newcommand{\ImportTok}[1]{#1}
\newcommand{\CommentTok}[1]{\textcolor[rgb]{0.56,0.35,0.01}{\textit{#1}}}
\newcommand{\DocumentationTok}[1]{\textcolor[rgb]{0.56,0.35,0.01}{\textbf{\textit{#1}}}}
\newcommand{\AnnotationTok}[1]{\textcolor[rgb]{0.56,0.35,0.01}{\textbf{\textit{#1}}}}
\newcommand{\CommentVarTok}[1]{\textcolor[rgb]{0.56,0.35,0.01}{\textbf{\textit{#1}}}}
\newcommand{\OtherTok}[1]{\textcolor[rgb]{0.56,0.35,0.01}{#1}}
\newcommand{\FunctionTok}[1]{\textcolor[rgb]{0.00,0.00,0.00}{#1}}
\newcommand{\VariableTok}[1]{\textcolor[rgb]{0.00,0.00,0.00}{#1}}
\newcommand{\ControlFlowTok}[1]{\textcolor[rgb]{0.13,0.29,0.53}{\textbf{#1}}}
\newcommand{\OperatorTok}[1]{\textcolor[rgb]{0.81,0.36,0.00}{\textbf{#1}}}
\newcommand{\BuiltInTok}[1]{#1}
\newcommand{\ExtensionTok}[1]{#1}
\newcommand{\PreprocessorTok}[1]{\textcolor[rgb]{0.56,0.35,0.01}{\textit{#1}}}
\newcommand{\AttributeTok}[1]{\textcolor[rgb]{0.77,0.63,0.00}{#1}}
\newcommand{\RegionMarkerTok}[1]{#1}
\newcommand{\InformationTok}[1]{\textcolor[rgb]{0.56,0.35,0.01}{\textbf{\textit{#1}}}}
\newcommand{\WarningTok}[1]{\textcolor[rgb]{0.56,0.35,0.01}{\textbf{\textit{#1}}}}
\newcommand{\AlertTok}[1]{\textcolor[rgb]{0.94,0.16,0.16}{#1}}
\newcommand{\ErrorTok}[1]{\textcolor[rgb]{0.64,0.00,0.00}{\textbf{#1}}}
\newcommand{\NormalTok}[1]{#1}

% Table formatting
\usepackage{longtable, booktabs}
\usepackage{lscape}
% \usepackage[counterclockwise]{rotating}   % Landscape page setup for large tables
\usepackage{multirow}		% Table styling
\usepackage{tabularx}		% Control Column width
\usepackage[flushleft]{threeparttable}	% Allows for three part tables with a specified notes section
\usepackage{threeparttablex}            % Lets threeparttable work with longtable

% Create new environments so endfloat can handle them
% \newenvironment{ltable}
%   {\begin{landscape}\begin{center}\begin{threeparttable}}
%   {\end{threeparttable}\end{center}\end{landscape}}

\newenvironment{lltable}
  {\begin{landscape}\begin{center}\begin{ThreePartTable}}
  {\end{ThreePartTable}\end{center}\end{landscape}}

  \usepackage{ifthen} % Only add declarations when endfloat package is loaded
  \ifthenelse{\equal{\string man}{\string man}}{%
   \DeclareDelayedFloatFlavor{ThreePartTable}{table} % Make endfloat play with longtable
   % \DeclareDelayedFloatFlavor{ltable}{table} % Make endfloat play with lscape
   \DeclareDelayedFloatFlavor{lltable}{table} % Make endfloat play with lscape & longtable
  }{}%



% The following enables adjusting longtable caption width to table width
% Solution found at http://golatex.de/longtable-mit-caption-so-breit-wie-die-tabelle-t15767.html
\makeatletter
\newcommand\LastLTentrywidth{1em}
\newlength\longtablewidth
\setlength{\longtablewidth}{1in}
\newcommand\getlongtablewidth{%
 \begingroup
  \ifcsname LT@\roman{LT@tables}\endcsname
  \global\longtablewidth=0pt
  \renewcommand\LT@entry[2]{\global\advance\longtablewidth by ##2\relax\gdef\LastLTentrywidth{##2}}%
  \@nameuse{LT@\roman{LT@tables}}%
  \fi
\endgroup}


  \usepackage{graphicx}
  \makeatletter
  \def\maxwidth{\ifdim\Gin@nat@width>\linewidth\linewidth\else\Gin@nat@width\fi}
  \def\maxheight{\ifdim\Gin@nat@height>\textheight\textheight\else\Gin@nat@height\fi}
  \makeatother
  % Scale images if necessary, so that they will not overflow the page
  % margins by default, and it is still possible to overwrite the defaults
  % using explicit options in \includegraphics[width, height, ...]{}
  \setkeys{Gin}{width=\maxwidth,height=\maxheight,keepaspectratio}
\ifxetex
  \usepackage[setpagesize=false, % page size defined by xetex
              unicode=false, % unicode breaks when used with xetex
              xetex]{hyperref}
\else
  \usepackage[unicode=true]{hyperref}
\fi
\hypersetup{breaklinks=true,
            pdfauthor={},
            pdftitle={Using Beta Regression to Model Normative Aspects of Prejudice and Political Attitudes: With Applications in R},
            colorlinks=true,
            citecolor=blue,
            urlcolor=blue,
            linkcolor=black,
            pdfborder={0 0 0}}
\urlstyle{same}  % don't use monospace font for urls

\setlength{\parindent}{0pt}
%\setlength{\parskip}{0pt plus 0pt minus 0pt}

\setlength{\emergencystretch}{3em}  % prevent overfull lines

\ifxetex
  \usepackage{polyglossia}
  \setmainlanguage{}
\else
  \usepackage[english]{babel}
\fi

% Manuscript styling
\captionsetup{font=singlespacing,justification=justified}
\usepackage{csquotes}
\usepackage{upgreek}



\usepackage{tikz} % Variable definition to generate author note

% fix for \tightlist problem in pandoc 1.14
\providecommand{\tightlist}{%
  \setlength{\itemsep}{0pt}\setlength{\parskip}{0pt}}

% Essential manuscript parts
  \title{Using Beta Regression to Model Normative Aspects of Prejudice and
Political Attitudes: With Applications in R}

  \shorttitle{Beta Regression}


  \author{Mark H. White II\textsuperscript{1}}

  \def\affdep{{""}}%
  \def\affcity{{""}}%

  \affiliation{
    \vspace{0.5cm}
          \textsuperscript{1} University of Kansas  }

  \authornote{
    \newcounter{author}
    Author note will go here.

                      Correspondence concerning this article should be addressed to Mark H. White II. E-mail: \href{mailto:markhwhiteii@gmail.com}{\nolinkurl{markhwhiteii@gmail.com}}
                }


  \abstract{Abstract will go here.}
  \keywords{beta regression, hurdle models, gamlss, norms, social attitudes \\

    
  }





\usepackage{amsthm}
\newtheorem{theorem}{Theorem}
\newtheorem{lemma}{Lemma}
\theoremstyle{definition}
\newtheorem{definition}{Definition}
\newtheorem{corollary}{Corollary}
\newtheorem{proposition}{Proposition}
\theoremstyle{definition}
\newtheorem{example}{Example}
\theoremstyle{remark}
\newtheorem*{remark}{Remark}
\begin{document}

\maketitle

\setcounter{secnumdepth}{0}



\section{Statistical Background}\label{statistical-background}

\subsection{The Beta Distribution}\label{the-beta-distribution}

The beta distribution can be used in a generalized linear model when the
values of the dependent variable are bounded \(0 < y < 1\) (Coxe, West,
\& Aiken, 2013). The probability density function (pdf) of the beta
distribution is determined by two parameters, \(\alpha\) and \(\beta\),
that are called \enquote{shape} parameters:

\begin{center}
$f(y;\alpha,\beta)={\Gamma(\alpha+\beta)\over\Gamma(\alpha)\Gamma(\beta)}y^{\alpha-1}(1-y)^{\beta-1}$
\end{center}

where \(\Gamma(.)\) is the gamma function. One of the benefits of the
beta distribution is that it is flexible and can take a number of
distributional shapes (Figure 1).

\begin{figure}
\centering
\includegraphics{beta_hurdle_files/figure-latex/unnamed-chunk-2-1.pdf}
\caption{\label{fig:unnamed-chunk-2}Beta probability density functions with
various combinations of shape parameters.}
\end{figure}

These parameters are not inherently meaningful to researchers, however.
R. Rigby, Stasinopoulos, Heller, and De Bastiani (2017)
\enquote{reparameterized} the beta distribution so that the two
parameters determining the shape of the distribution would be more
useful in a regression framework (but see Ferrari \& Cribari-Neto, 2004
for a different parameterization). Instead of predicting \(\alpha\) and
\(\beta\), they parameterize (i.e., algebraically rearrange parameters)
so that beta regression predicts two different parameters: \(\mu\)
(called the \enquote{location} parameter) and \(\sigma\) (called the
\enquote{scale} parameter), where \(\mu={\alpha\over(\alpha+\beta)}\)
and \(\sigma=\sqrt{1\over(\alpha+\beta+1)}\). \(\mu\) is equivalent to
the mean, and \(\sigma\) is related positively to the variance (note
that \(\sigma\) is \emph{not} the standard deviation, even though
\(\sigma\) commonly refers to the standard deviation). The variance is
equivalent to \(\sigma^2\mu(1-\mu)\), and there are two things to note
from this equation: First, the greater the \(\sigma\), the greater the
variance; second, the variance depends on the mean, which means that
beta regression will be naturally heteroskedastic (covered shortly).
Using this distribution in a regression framework cannot handle the
dependent variable taking on values 0 and 1---how can we model dependent
variables in the range \(0 \leq y \leq 1\)?

\subsection{Zero-One Inflated Beta
Distribution}\label{zero-one-inflated-beta-distribution}

R. Rigby et al. (2017) show that the beta distribution can be
\enquote{inflated} at 0 and 1, meaning the dependent variable contains
0s and 1s (i.e., \(0 \leq y \leq 1\)), the pdf for this beta inflated
distribution, \(\text{beinf}\), is:

\begin{center}
\[
\text{beinf}(y;\mu,\sigma,\nu,\tau) =
\begin{cases}
  p_0                             & \text{if } y = 0\\
  (1 - p_0 - p_1)f(y;\mu,\sigma)  & \text{if } 0 < y < 1\\
  p_1                             & \text{if } y = 1
\end{cases}
\]
\end{center}

for \(0 \leq y \leq 1\), where \(f(y;\mu,\sigma)\) is the beta pdf with
\(\mu\) and \(\sigma\) bounded \emph{between} zero and one. The two
additional parameters, \(\nu\) and \(\tau\), are related to \(p_0\) and
\(p_1\), respectively. \(p_0\) is the probability that a case equals
zero, \(p_1\) is the probability that a case equals one, and \(p_2\)
(i.e., \(1 - p_0 - p_1\)) is the probability that the case comes from
the beta distribution. In terms of these two additional parameters,
\(p_0 = {\nu \over (1 + \nu + \tau)}\) and
\(p_1 = {\tau \over (1 + \nu + \tau)}\). Rearranging these
algebraically, \(\nu\) is the odds that a case is 0 to being
\(0 < y < 1\), \(\nu = p_0 / p_2\), and \(\tau\) is the odds that a case
is a 1 to being \(0 < y < 1\), \(\tau = p_1 / p_2\).

\subsection{Beta Regression Models}\label{beta-regression-models}

The goal for a researcher now is to predict these four parameters,
\(\mu, \sigma, \nu, \tau\), from any number of predictor variables of
theoretical interest. All four parameters can be predicted by an
identical set of predictors, none of the same predictors, or any
mixture. Both \(\mu\) and \(\sigma\) have to be between zero and one
(since the beta distribution is bounded between 0 and 1), so one can use
the logistic link function to fit predicted values in this range. Both
\(\nu\) and \(\tau\) have to be greater than zero (since they are odds),
so one can use the log link function to fit predicted values in this
range. Imagine a researcher has one predictor of interest, \(x\). They
could include this variable as a predictor of all four variables with
the equations:

\begin{center}
$\text{log}({\mu \over 1 - \mu}) = \beta_{10} + \beta_{11}X$\\
$\text{log}({\sigma \over 1 - \sigma}) = \beta_{20} + \beta_{21}X$\\
$\text{log}(\nu) = \beta_{30} + \beta_{31}X$\\
$\text{log}(\tau) = \beta_{40} + \beta_{41}X$
\end{center}

Or equivalently:

\begin{center}
$\mu={1\over1+e^{-(\beta_{10}+\beta_{11}X)}}$\\
$\sigma={1\over1+e^{-(\beta_{20}+\beta_{21}X)}}$\\
$\nu = e^{\beta_{30} + \beta_{31}X}$\\
$\tau = e^{\beta_{40} + \beta_{41}X}$
\end{center}

where \(e^x\) is the natural exponential function. A different inflated
beta distribution can also be used when the dependent variable contains
0s but not 1s (e.g., \(0 \leq y < 1\)) or when it contains 1s but not 0s
(e.g., \(0 < y \leq 1\)). Let \(c\) be the value---0 or 1---that is
included (i.e., inflated). The pdf is:

\begin{center}
\[
\text{beinf}_c(y;\mu,\sigma,\nu) =
\begin{cases}
  p_c                             & \text{if } y = c\\
  (1 - p_c)f(y;\mu,\sigma)        & \text{if } 0 < y < 1\\
\end{cases}
\]
\end{center}

where \(\nu = p_c / (1 - p_c)\) and the remaining notation is the same
as in the zero-and-one inflated beta distribution. The same link
functions are used, but the fourth parameter, \(\tau\), is not included,
since the distribution is only inflated at \(c\). Lastly, if no 0s or 1s
are observed, the beta distribution alone can be used as the pdf. This
results in a beta regression where the researcher is only predicting the
location, \(\mu\), and shape, \(\sigma\). Since there is no inflation at
0 or 1, the latter two parameters, \(\nu\) and \(\tau\), are not
included.

\subsection{Modeling Bounded Variables Beyond the Zero Through One
Range}\label{modeling-bounded-variables-beyond-the-zero-through-one-range}

These beta regression models are commonly used when rates and
proportions are dependent variables, given that these are naturally
bounded \(0 \leq y \leq 1\) (Buntaine, 2011; Eskelson, Madsen, Hagar, \&
Temesgen, 2011; Gallardo, Bovea, Colomer, \& Prades, 2012; Hubben et
al., 2008; Peplonska et al., 2012). More generally, however, the beta
distribution is a doubly bounded continuous distribution: Although it is
on a continuous scale, values cannot be exceed the upper bound, \(u\),
or be less than the lower bound, \(l\). In the case of the zero-and-one
inflated beta distribution, \(l = 0\) and \(u = 1\).

Most scales used to measure prejudice and political attitudes are doubly
bounded and continuous. Although researchers generally model variables
on Likert scales and sliding scales (e.g., feeling thermometers) as
being conditionally normally distributed (i.e., using ordinary least
squares regression), these variables are not strictly normal.
Observations from a normal distribution can take on any value on the
real number line, while observations from a standard 7-point Likert
scale can only take on values 1 through 7---and in studying
controversial and polarizing issues like prejudice and politics, many
participants score at the lower or upper bounds, causing floor or
ceiling effects. In these cases, the assumptions of normality and
homoskedasticity are likely to be violated. Many times researchers
create, update, or seek out measures to so that the distributional form
of the dependent variable takes on a normal shape---statistical
assumptions are directing the phenomenon researchers choose to observe
and the theoretical questions that they ask. Instead, one can explicitly
take into account that the response is doubly bounded and
heteroskedastic by using the beta regression models described above,
using a simple linear rescaling of the data.

If one observes a dependent variable limited between two bounds, there
is a straightforward way to rescale the variable to the
\(0 \leq y \leq 1\) range:

\begin{center}
$y'_i = (y_i - l) / (u - l)$
\end{center}

where \(y\) is the variable on the original scale, \(y'\) is the
rescaled variable, \(u\) is the upper bound (i.e., the largest possible
value on the scale), \(l\) is the lower bound (i.e., the smallest
possible value on the scale), and the \(i\) subscript denotes an
individual's score. On a standard 7-point Likert scale, \(l = 1\) and
\(u = 7\). This rescaling allows a researcher to explicitly model
conditional variance, floor effects, and ceiling effects using beta
regression.

\subsubsection{Conditional variance}\label{conditional-variance}

Observing a dependent variable that is doubly bounded and continuous can
produce heteroskedasticity. One of the assumptions of an OLS regression
is homoskedasticity---that the variance of the errors are unrelated to
any predictor, any linear combination of predictors, and the predicted
values. A regression equation with one predictor variable \(x\) is often
written as \(y_i = \beta_0 + \beta_1x_i + \epsilon_i\), where
\(\epsilon_i\) is how far an osbserved value \(y_i\) is from the
predicted value (i.e., the residual). Let \(\hat{y_i}\) be the predicted
value for observation \(i\), where \(\hat{y_i} = \beta_0 + \beta_1x_i\).
We can simplify the equation to \(y_i = \hat{y_i} + \epsilon_i\), where
the residuals \(\epsilon\) are normally distributed with a mean of 0 and
some variance---that is, \(\epsilon \sim N(0, \sigma^2_\epsilon)\). We
can further simplify this to:
\(y_i|x_i \sim N(\hat{y_i}, \sigma^2_\epsilon)\), which means that each
observation of the dependent variable we make, given the predictors we
use, is normally distributed with a mean of that observation's predicted
score and some variance. The assumption of homoskedasticity is found in
that \(\sigma^2_\epsilon\) does \emph{not} have a subscript \(i\). This
means that there is one \emph{common variance} of the residuals.

Imagine one runs a regression and observes
\(\hat{y_i} = 2.5 + 1.5 \times x_i\) and \(\sigma^2_\epsilon = 9\). Say
that the first individual's score on \(x\) is 0, \(x_1 = 0\), and the
second individual's score is \(x_2 = 5\). The model then assumes that
the first person comes from a normal distribution with a mean of 4
(i.e., \(\hat{y_1}\)) and a variance of 9 (i.e., \(\sigma^2_\epsilon\)),
while the second person comes from a normal distribution with a mean of
10 and that same variance of 9. The violation of this assumption can
lead to inflated Type I errors or diminished statistical power,
depending on the type and source of the heteroskedasticity (Hayes \&
Cai, 2007; Long \& Ervin, 2000; Rosopa, Schaffer, \& Schroeder, 2013).

While common remedies for heteroskedasticity (e.g., robust standard
errors, transforming the dependent variable, weighted least squares)
treat it as a nuisance to be corrected for when calculating \(p\)-values
(Hayes \& Cai, 2007; Long \& Ervin, 2000; Rosopa et al., 2013), it could
be that explicitly modeling this conditional variance is an interesting
phenomenon per se. Heteroskedasticity can arise when the error variance
is predicted by one of the observed predictor variables, and since beta
regression models predict a shape parameter, \(\sigma\), a researcher
can explicitly model this relationship. This allows researchers to ask
questions about conditional variance: \enquote{Does the variance in
\(y\) increase as \(x\) also increases?} Since the focus of OLS
regression is the \emph{location} parameter (e.g., conditional mean),
researchers might be ignoring interesting, potentially
theoretically-relevant parts of their data. After discussing floor and
ceiling effects, I will argue that normative influences are one of these
aspects.

\subsubsection{Floor and ceiling
effects}\label{floor-and-ceiling-effects}

Floor and ceiling effects occur when there are many observations of the
dependent variable at the lower and upper bound of the scale,
respectively (Everitt \& Skrondal, 2010). Variables displaying these
effects are likely to violate numerous assumptions of ordinary least
squares regression, such as conditional normality and homoskedasticity
of variance, and can be modeled as censored or truncated. Censoring
occurs when all scores beyond some threshold are recorded at that
threshold. For example, if a researcher is measuring reaction times,
censoring occurs if they decide that any times above 5 seconds are
scored as 5. If many people took longer than 5 seconds, this censoring
can cause a ceiling effect. Truncating occurs when people who score
beyond some treshold are excluded from the data. If the researcher
simply does not record trials that take longer than 5 seconds, then the
data are truncated. A common regression technique for censored or
truncated variables are Tobit models (McBee, 2010; Smithson \& Merkle,
2013). However, standard Tobit models assume that the underlying latent
construct is normally distributed.

In the realm of prejudice and politics, I argue that it is \emph{not}
useful to think of prejudice as coming from a latent normal
distribution. I find it unlikely that if, instead of a Likert scales
going from 1 to 7, it extended from -7 to +7, one would observe a normal
distribution for the Attitude Towards Blacks scale, for example (Brigham
(1993)). I find it likely that participants would opt for the new
floor---the negative 7. This can be thought of as a decision-making
process. Using prejudice as an example, people decide whether or not
they are going to admit to feeling any prejudice. If they decide not to,
they simply circle or click all 1s (i.e., the floor). If they decide to
admit prejudice, they respond to the items. As another example: When
offering one's attitudes toward a polarizing political figure, a
participant decides whether or not to respond entirely in the negative
(i.e., at the floor), entirely in the positive (i.e., at the ceiling),
or somewhere in between. This could result in a bimodal distribution,
with both ceiling and floor effects.

This decision making process can be modeled with the inflated beta
regression models described above. Inflations at zero, one, or
zero-and-one allow researchers to measuring ceiling effects, floor
effects, and both simultaneously. While methodologists in beta
regression refer to these models as \enquote{inflated} models, a more
general label is \enquote{two-part} models (Coxe et al., 2013). When one
assumes a decision-making process behind the two parts, these models are
often referred to as \enquote{hurdle} models in the econometrics
literature (Cameron \& Trivedi, 2005; Wooldridge, 2010).

Cragg (1971) demonstrated that hurdle models are a generalization of the
Type I Tobit model (but note that, unlike the standard Tobit model, the
latent dependent construct is assumed to be conditionally \emph{beta}
distributed, which can take on many distributional forms; see Figure 1).
In economics, hurdle models are often used when dependent variables are
counts with excess 0s (see Carlevaro, Croissant, \& Hoareau, 2017 for a
review of applications). The log likelihood function for these models
can be separated in two parts: First, a logistic (or probit) regression
modeling the probability of observing \(y = 0\) versus \(y > 0\);
second, a truncated negative binomial model using just the cases where
\(y > 0\).

The inflated beta regressions can be seen as hurdle models, as the log
likelihood for the zero-and-one inflated beta regression can also be
separated in two parts: First, a multinomial logistic regression
modeling the probabilities of observing a \(y = 0\), \(y = 1\), or
\(0 < y < 1\); second, a beta regression model using just the cases
where \(0 < y < 1\). Similarly, the inflated beta regression at
\emph{either} zero \emph{or} one has a log likelihood function that can
be separated in two parts: First, a logistic regression modeling the
probability of observing \(y = c\) or \(0 < y < 1\); second, a beta
regression modeling just the cases where \(0 < y < 1\) (R. Rigby et al.,
2017).

Since their likelihoods can be separated, this implies a critical
assumption: That the two processes are independent, given the observed
predictors. In other words, the error terms of each submodel are assumed
to be independent of one another (where submodels refer to models
predicting each parameter, \(\mu\), \(\sigma\), \(\nu\), or \(\tau\),
separately). Dependencies between them can be modeled (referred to as
``selection'' models; Carlevaro et al., 2017; Wooldridge, 2010), but I
will not discuss these models here, as they are not extended to beta
regression.

\section{Norms Produce Invariance, Ceiling and Floor
Effects}\label{norms-produce-invariance-ceiling-and-floor-effects}

Louis, Mavor, and Terry (2003) argue that norms produce invariance. In
this section, I demonstrate the relationship between normative strength
and invariance of attitudes empirically. I replicated the methodology
used by Crandall, Eshleman, and O'brien (2002). In their Study 1, the
authors presented participants with 105 social groups (e.g., Black
people, librarians, deaf people, ugly people, Nazis, drug dealers).
Participants were assigned to one of two groups. Half of the
participants indicated how negatively they felt toward the group (i.e.,
prejudice), while the other half indicated how socially acceptable they
perceived it was to feel and express negativity toward that group (i.e.,
acceptability). Crandall and colleagues averaged the prejudice and
acceptability scores at the group level, resulting in an \(N = 105\) and
two variables for each group: prejudice and acceptability. They found
that expressed prejudice and acceptability of prejudice correlated at
\(r = .96\); people report prejudices that are socially acceptable (see
Cary and Page-Gould (2014) for a replication).

I recruited 405 people from Amazon's Mechanical Turk website and
randomly assigned them to either a \emph{prejudice} or
\emph{acceptability} condition. All participants answered questions
about 30 groups, randomly sampled from a pool of 120 groups. In the
prejudice condition, participants were asked \enquote{How do you feel
toward each of these groups?} and to indicate how they feel on a scale
from \enquote{very negatively (0) to very positively (100).} I
reverse-scored these items so that higher scores indicated more
prejudice. In the acceptability condition, participants were asked
\enquote{How OK is it in society for people to feel and say negative
things about each of these groups?} and to indicate their perception on
a scale from \enquote{totally unacceptable (0) to completely acceptable
(100) for people to feel and say negative things about these groups.}

To demonstrate the relationship between norms and invariance, I
calculated the median social acceptability score and the variance of the
self-reported prejudice scores for each of the 120 groups. There was no
linear relationship between social acceptability and variance of
reported prejudice, \(b = 0.31, SE = 0.84, t(118) = 0.37, p = .714\);
however, there was a quadratic relationship regressing variance of
reported prejudice on median social acceptability,
\(\Delta R^2 = .28, b = -1250.07, SE = 187.44, t(117) = -6.67, p < .001\)
(Figure 2). The closer a prejudice was perceived to be by some
participants as totally unacceptable (0) or compeltely acceptable (100),
the lower was the variance of self-reported prejudice by other
participants.

\begin{figure}
\centering
\includegraphics{beta_hurdle_files/figure-latex/unnamed-chunk-3-1.pdf}
\caption{\label{fig:unnamed-chunk-3}The stronger the normative influence on
prejudice, the smaller the variance of reported prejudice.}
\end{figure}

I also investigated normative influences on floor effects. Crandall
(1994) proposed a measure for anti-fat prejudice. In Study 5 of this
paper, Crandall makes the argument that anti-fat attitudes are more
socially acceptable (and less subject to social desirability biases)
than anti-Black prejudice. He does this by demonstrating the
decision-making process described above in the hurdle model: How often
do people \emph{always} select the least prejudiced option? That is, how
often do people respond \emph{always} at the lower bound of the
distribution? He calls the proportion of people at the floor of a scale
the \enquote{politically correct (PC) index,} as this shows how often
people respond in the most socially acceptable (i.e., politically
correct) way possible. He finds that anti-fat prejudice is less subject
to these floor effects: 10\% of participants always responded at the
lower bound for anti-Black prejudice; this was only 3\% for Crandall's
anti-fat scale.

To show that norms produce floor effects, I calculated Crandall's PC
index for each of the 120 groups by counting how many people responded
with a 0 on the prejudice thermometer (i.e., \enquote{totally positive}
feelings). Using a negative binomial regression (Venables \& Ripley,
2002) for these count data, I regressed the PC index on median social
acceptability of the prejudice. Perceived social acceptability of the
prejudice by one group of participants negatively predicted the PC
index, calculated using a separate group of participants,
\(b = -0.03, SE = 0.002, Z = -10.97, p < .001\) (Figure 3). The more
socially acceptable a prejudice, the less people opt for the
\enquote{politically correct} floor of the scale.

\begin{figure}
\centering
\includegraphics{beta_hurdle_files/figure-latex/unnamed-chunk-4-1.pdf}
\caption{\label{fig:unnamed-chunk-4}The more socially acceptable the
prejudice, the less people opt for the politically correct response.}
\end{figure}

This evidence shows that norms do positively predict invariance;
predicting the scale parameter, \(\sigma\), in beta regression could be
used to investigate how much people adhere to norms regarding the
expression of certain attitudes. These data also show that normative
influence can lead to floor effects (or ceiling effects); predicting the
shape parameters, \(\nu\) and \(\tau\), in beta regression could also be
used to examine how much people adhere to norms. I now turn to
demonstrating how these beta regression models can be used in R (R Core
Team, 2017) via the \texttt{gamlss} package (R. A. Rigby \&
Stasinopoulos, 2005).

\section{Implementation in GAMLSS}\label{implementation-in-gamlss}

The \texttt{gamlss} name stands for generalized additive models for
location (i.e., \(\mu\)), scale (i.e., \(\sigma\)), and shape (i.e.,
\(\nu\) and \(\tau\)). This package gives researchers the ability to use
a wide variety of models, including beta regression models. A number of
algorithms to estimate the coefficients can be used, described in detail
by M. D. Stasinopoulos, Rigby, Heller, Voudouris, and De Bastiani
(2017). To first demonstrate that the package works according to the
parameterization discussed in the Statistical Background section, I will
generate a zero-and-one inflated beta distribution and estimate the
parameters with an intercepts-only regression using the \texttt{gamlss}
function. Then, I will demonstrate how to model social attitudes with
real data.

\subsection{Intercepts-Only
Regression}\label{intercepts-only-regression}

We can set a number of parameters for this zero-one-inflated regression
using the following code:

\begin{Shaded}
\begin{Highlighting}[]
\NormalTok{n <-}\StringTok{ }\DecValTok{5000}
\NormalTok{mu <-}\StringTok{ }\FloatTok{0.40}
\NormalTok{sigma <-}\StringTok{ }\FloatTok{0.60}
\NormalTok{p0 <-}\StringTok{ }\FloatTok{0.13}
\NormalTok{p1 <-}\StringTok{ }\FloatTok{0.17}
\NormalTok{p2 <-}\StringTok{ }\DecValTok{1} \OperatorTok{-}\StringTok{ }\NormalTok{p0 }\OperatorTok{-}\StringTok{ }\NormalTok{p1}
\NormalTok{a <-}\StringTok{ }\NormalTok{mu }\OperatorTok{*}\StringTok{ }\NormalTok{(}\DecValTok{1} \OperatorTok{-}\StringTok{ }\NormalTok{sigma }\OperatorTok{^}\StringTok{ }\DecValTok{2}\NormalTok{) }\OperatorTok{/}\StringTok{ }\NormalTok{(sigma }\OperatorTok{^}\StringTok{ }\DecValTok{2}\NormalTok{)}
\NormalTok{b <-}\StringTok{ }\NormalTok{a }\OperatorTok{*}\StringTok{ }\NormalTok{(}\DecValTok{1} \OperatorTok{-}\StringTok{ }\NormalTok{mu) }\OperatorTok{/}\StringTok{ }\NormalTok{mu}
\end{Highlighting}
\end{Shaded}

The sample size \texttt{n} \(= 5000\), the mean \texttt{mu} \(= 0.40\),
the shape parameter \texttt{sigma} \(= 0.60\), the probability of being
0 is \texttt{p0} \(= 0.13\), the probability of being 1 is \texttt{p1}
\(= 0.17\), and the probability of being between 0 and 1 is \texttt{p2}
\(= 1 - p_0 - p_1 = .70\). Using the equations in the Statistical
Background section, we can convert \texttt{mu} and \texttt{sigma} back
to the original shape parameters \texttt{a} (\(\alpha\)) and \texttt{b}
(\(\beta\)). The dependent variable \texttt{y} can now be generated
using the following code:

\begin{Shaded}
\begin{Highlighting}[]
\KeywordTok{set.seed}\NormalTok{(}\DecValTok{1839}\NormalTok{)}
\NormalTok{y <-}\StringTok{ }\KeywordTok{vector}\NormalTok{(}\StringTok{"numeric"}\NormalTok{, n)}
\ControlFlowTok{for}\NormalTok{ (i }\ControlFlowTok{in} \DecValTok{1}\OperatorTok{:}\NormalTok{n) \{}
\NormalTok{  rand <-}\StringTok{ }\KeywordTok{runif}\NormalTok{(}\DecValTok{1}\NormalTok{)}
  \ControlFlowTok{if}\NormalTok{ (rand }\OperatorTok{<=}\StringTok{ }\NormalTok{p0) \{}
\NormalTok{    y[i] <-}\StringTok{ }\DecValTok{0}
\NormalTok{  \} }\ControlFlowTok{else} \ControlFlowTok{if}\NormalTok{ ((p0 }\OperatorTok{<}\StringTok{ }\NormalTok{rand) }\OperatorTok{&}\StringTok{ }\NormalTok{(rand }\OperatorTok{<=}\StringTok{ }\NormalTok{p0 }\OperatorTok{+}\StringTok{ }\NormalTok{p1)) \{}
\NormalTok{    y[i] <-}\StringTok{ }\DecValTok{1}
\NormalTok{  \} }\ControlFlowTok{else}\NormalTok{ \{}
\NormalTok{    y[i] <-}\StringTok{ }\KeywordTok{rbeta}\NormalTok{(}\DecValTok{1}\NormalTok{, a, b)}
\NormalTok{  \}}
\NormalTok{\}}
\end{Highlighting}
\end{Shaded}

I generate a random number, \texttt{rand}, between 0 to 1. If
\texttt{rand} \(\leq\) \texttt{p0}, then the case is 0; if \texttt{rand}
\(>\) \texttt{p0} and \texttt{rand} \(\leq\) \texttt{p0\ +\ p1}, then
the case is 1; otherwise, the case comes from a beta distribution with
shape parameters of \texttt{a} and \texttt{b}. The \texttt{gamlss}
function has four arguments for formulas---one for each of the
parameters. It also has an argument, \texttt{family}, that determines
what distribution will be used in fitting the model. For a zero-and-one
inflated beta regression, the family is \texttt{BEINF()}. One can fit
the intercepts-only model using the following code:

\begin{Shaded}
\begin{Highlighting}[]
\NormalTok{fit <-}\StringTok{ }\KeywordTok{gamlss}\NormalTok{(}
  \DataTypeTok{formula =}\NormalTok{ x }\OperatorTok{~}\StringTok{ }\DecValTok{1}\NormalTok{, }\CommentTok{# formula for mu}
  \DataTypeTok{formula.sigma =} \OperatorTok{~}\StringTok{ }\DecValTok{1}\NormalTok{, }\CommentTok{# formula for sigma}
  \DataTypeTok{formula.nu =} \OperatorTok{~}\StringTok{ }\DecValTok{1}\NormalTok{, }\CommentTok{# formula for nu}
  \DataTypeTok{formula.tau =} \OperatorTok{~}\StringTok{ }\DecValTok{1}\NormalTok{, }\CommentTok{# formula for tau}
  \DataTypeTok{family =} \KeywordTok{BEINF}\NormalTok{() }\CommentTok{# distribution for model}
\NormalTok{)}
\end{Highlighting}
\end{Shaded}

We can use the inverse link functions to transform the coefficients back
into the original scale. Then we can compare it to the parameters we set
above. The estimated parameters are estimated using the code, according
to the equations and link functions described in the Statistical
Background section:

\begin{Shaded}
\begin{Highlighting}[]
\NormalTok{inv_logit <-}\StringTok{ }\ControlFlowTok{function}\NormalTok{(x) }\KeywordTok{exp}\NormalTok{(x) }\OperatorTok{/}\StringTok{ }\NormalTok{(}\DecValTok{1} \OperatorTok{+}\StringTok{ }\KeywordTok{exp}\NormalTok{(x)) }\CommentTok{# inverse of link function}
\NormalTok{fit_mu <-}\StringTok{ }\KeywordTok{inv_logit}\NormalTok{(fit}\OperatorTok{$}\NormalTok{mu.coefficients)}
\NormalTok{fit_sigma <-}\StringTok{ }\KeywordTok{inv_logit}\NormalTok{(fit}\OperatorTok{$}\NormalTok{sigma.coefficients)}
\NormalTok{fit_nu <-}\StringTok{ }\KeywordTok{exp}\NormalTok{(fit}\OperatorTok{$}\NormalTok{nu.coefficients)}
\NormalTok{fit_tau <-}\StringTok{ }\KeywordTok{exp}\NormalTok{(fit}\OperatorTok{$}\NormalTok{tau.coefficients)}
\NormalTok{fit_p0 <-}\StringTok{ }\NormalTok{fit_nu }\OperatorTok{/}\StringTok{ }\NormalTok{(}\DecValTok{1} \OperatorTok{+}\StringTok{ }\NormalTok{fit_nu }\OperatorTok{+}\StringTok{ }\NormalTok{fit_tau)}
\NormalTok{fit_p1 <-}\StringTok{ }\NormalTok{fit_tau }\OperatorTok{/}\StringTok{ }\NormalTok{(}\DecValTok{1} \OperatorTok{+}\StringTok{ }\NormalTok{fit_nu }\OperatorTok{+}\StringTok{ }\NormalTok{fit_tau)}
\end{Highlighting}
\end{Shaded}

The estimates for \(\mu\), \(\sigma\), \(p_0\), and \(p_1\) are \(.41\),
\(.60\), \(.13\), and \(.17\), respectively. These are close estimates
to the population parameters set above. I now turn to a model using real
data.

\subsection{Modeling Political
Ideology}\label{modeling-political-ideology}

The social dominance orientation (SDO) scale measures how much one
supports social hierarchy and inequality in society (Ho et al., 2015;
Pratto, Sidanius, Stallworth, \& Malle, 1994). In personality
psychology, it has a robust, oft-replicated positive relationship with
political conservatism (Duriez \& Van Hiel, 2002; Hiel \& Mervielde,
2002; Ho et al., 2015; Pratto et al., 2000; Pratto, Stallworth, \&
Conway-Lanz, 1998; Pratto, Stallworth, \& Sidanius, 1997; Wilson \&
Sibley, 2013); some consider SDO a central aspect of contemporary
American conservatism (Eidelman, Crandall, Goodman, \& Blanchar, 2012;
Jost, Glaser, Kruglanski, \& Sulloway, 2003). However, openly admitting
that some groups are inherently better than others is a socially
unacceptable thing to express, often leading to floor effects. I regress
SDO on conservatism in both an OLS and beta regression framework to
compare and contrast the two methods.

The data come from Study 1 of White II and Crandall (2017), where 175
participants answered an 8-item SDO scale on a 7-point Likert scale (Ho
et al., 2015) and two 7-point items that assessed conservatism (i.e.,
identification from Democrat to Republican and liberal to conservative,
averaged together). Figure 4 shows the histogram of the SDO scale---30\%
of the sample responded with 1s for each of the 8 items, demonstrating a
floor effect.

\begin{figure}
\centering
\includegraphics{beta_hurdle_files/figure-latex/unnamed-chunk-11-1.pdf}
\caption{}
\end{figure}

The two variables correlated at \(r = .50, p < .001\). Figure 5 shows
the predicted values against the standardized residuals from regressiong
SDO on conservatism using OLS, with a marginal histogram showing the
distribution of residuals. A loess line is drawn, and a slight jitter
has been added to the points, given that many points overlapped with one
another. One can see a spread of residuals getting wider as predicted
values increases, suggesting heteroskedasticity; moreover, the histogram
shows that the residuals are positively skewed, violating the assumption
of conditional normality. How does this look using beta regression?

\begin{figure}
\centering
\includegraphics{beta_hurdle_files/figure-latex/unnamed-chunk-12-1.pdf}
\caption{}
\end{figure}

First, I need to rescale the data. SDO was measured on a 1-to-7 scale,
so \(l = 1\) and \(u = 7\). One can use the following R code to rescale
a variable to \(0 \leq y \leq 1\), following the equation described in
the Modeling Bounded Variables Beyond the Zero Through One Range
section:

The last thing a researcher has to do is decide which argument to use
for the \texttt{family} argument in the \texttt{gamlss} function.
Researchers can follow this guide:

\begin{itemize}
\item
  If no 0s and no 1s are observed, a standard beta regression can be
  used. The \texttt{family} argument is \texttt{BE()}, and only
  \texttt{mu} and \texttt{sigma} will have submodels predicting these
  parameters.
\item
  If 0s are observed but 1s are not, a zero-inflated beta regression can
  be used. The \texttt{family} argument is \texttt{BEINF0()}, and
  \texttt{mu}, \texttt{sigma}, and \texttt{nu} will have submodels
  predicting these parameters.
\item
  If 0s are not observed but 1s are, a one-inflated beta regression can
  be used. The \texttt{family} argument is \texttt{BEINF1()}, and
  \texttt{mu}, \texttt{sigma}, and \texttt{nu} will have submodels
  predicting these parameters.
\item
  If 0s and 1s are observed, a zero-and-one inflated beta regression can
  be used. The \texttt{family} argument is \texttt{BEINF()}, and
  \texttt{mu}, \texttt{sigma}, \texttt{nu}, and \texttt{tau} will have
  submodels predicting these parameters.
\end{itemize}

One can check to see the minimum and maximum observed by using the
\texttt{range()} function in R. In this case, \texttt{range(dat\$sdo)}
returns \texttt{0\ 1}, showing that both 0s and 1s are observed. Now, I
run the beta regression model using the following code:

\begin{Shaded}
\begin{Highlighting}[]
\NormalTok{fit_beinf <-}\StringTok{ }\KeywordTok{gamlss}\NormalTok{(sdo }\OperatorTok{~}\StringTok{ }\NormalTok{rw_polid, }\OperatorTok{~}\StringTok{ }\NormalTok{rw_polid, }\OperatorTok{~}\StringTok{ }\NormalTok{rw_polid, }\OperatorTok{~}\StringTok{ }\NormalTok{rw_polid,}
                    \DataTypeTok{data =}\NormalTok{ dat, }\DataTypeTok{family =} \KeywordTok{BEINF}\NormalTok{())}
\end{Highlighting}
\end{Shaded}

Finally, one can use the \texttt{summary()} function to examine the
results. Table 1 shows the coefficients tables for each of the four
submodels from the summary.

\begin{table}[tbp]
\begin{center}
\begin{threeparttable}
\caption{\label{tab:unnamed-chunk-16}}
\begin{tabular}{llrrrr}
\toprule
Submodel & Variable & $b$ & $SE$ & $t$ & $p$\\
\midrule
$\mu$ & Intercept & -2.03 & 0.20 & -10.15 & < .001\\
 & Conservatism & 0.28 & 0.05 & 5.46 & < .001\\
$\sigma$ & Intercept & -0.97 & 0.22 & -4.42 & < .001\\
 & Conservatism & 0.14 & 0.06 & 2.48 & .014\\
$\nu$ & Intercept & 0.82 & 0.38 & 2.13 & .034\\
 & Conservatism & -0.54 & 0.12 & -4.46 & < .001\\
$\tau$ & Intercept & -13.33 & 6.91 & -1.93 & .055\\
 & Conservatism & 1.70 & 1.15 & 1.47 & .143\\
\bottomrule
\end{tabular}
\end{threeparttable}
\end{center}
\end{table}

Figure 6 shows the OLS and sigma of the residuals around it. It is
constant and goes into where observations cannot be made.

\begin{figure}
\centering
\includegraphics{beta_hurdle_files/figure-latex/unnamed-chunk-17-1.pdf}
\caption{}
\end{figure}

Figure 7 demonstrates the heteroskedasticity of the model, also showing
how variance increases as does conservatism.

\begin{figure}
\centering
\includegraphics{beta_hurdle_files/figure-latex/unnamed-chunk-18-1.pdf}
\caption{}
\end{figure}

\newpage

\section{References}\label{references}

\setlength{\parindent}{-0.5in} \setlength{\leftskip}{0.5in}

\hypertarget{refs}{}
\hypertarget{ref-brigham1993college}{}
Brigham, J. C. (1993). College students? Racial attitudes. \emph{Journal
of Applied Social Psychology}, \emph{23}(23), 1933--1967.

\hypertarget{ref-buntaine2011does}{}
Buntaine, M. T. (2011). Does the Asian Development Bank respond to past
environmental performance when allocating environmentally risky
financing? \emph{World Development}, \emph{39}(3), 336--350.

\hypertarget{ref-cameron2005microeconometrics}{}
Cameron, A. C., \& Trivedi, P. K. (2005). \emph{Microeconometrics:
Methods and applications}. New York, NY: Cambridge university press.

\hypertarget{ref-carlevaro2016multiple}{}
Carlevaro, F., Croissant, Y., \& Hoareau, S. (2017). \emph{Multiple
hurdle Tobit models in R: The mhurdle package}. Retrieved from
\url{https://cran.r-project.org/web/packages/mhurdle/vignettes/mhurdle.pdf}

\hypertarget{ref-cary2014prevalence}{}
Cary, L. A., \& Page-Gould, E. (2014). The prevalence and impact of
sexism, racism and homophobia in online gaming environments. Poster
presented at the annual meeting of the Society for Personality and
Social Psychology, Austin, TX.

\hypertarget{ref-coxe2013generalized}{}
Coxe, S., West, S. G., \& Aiken, L. S. (2013). Generalized linear
models. In T. D. Little (Ed.), \emph{The Oxford handbook of quantitative
methods, volume 2} (pp. 26--51). New York, NY: Oxford University Press.

\hypertarget{ref-cragg1971some}{}
Cragg, J. G. (1971). Some statistical models for limited dependent
variables with application to the demand for durable goods.
\emph{Econometrica: Journal of the Econometric Society}, 829--844.

\hypertarget{ref-crandall1994prejudice}{}
Crandall, C. S. (1994). Prejudice against fat people: Ideology and
self-interest. \emph{Journal of Personality and Social Psychology},
\emph{66}(5), 882.

\hypertarget{ref-crandall2002social}{}
Crandall, C. S., Eshleman, A., \& O'brien, L. (2002). Social norms and
the expression and suppression of prejudice: The struggle for
internalization. \emph{Journal of Personality and Social Psychology},
\emph{82}(3), 359.

\hypertarget{ref-duriez2002march}{}
Duriez, B., \& Van Hiel, A. (2002). The march of modern fascism: A
comparison of social dominance orientation and authoritarianism.
\emph{Personality and Individual Differences}, \emph{32}(7), 1199--1213.

\hypertarget{ref-eidelman2012low}{}
Eidelman, S., Crandall, C. S., Goodman, J. A., \& Blanchar, J. C.
(2012). Low-effort thought promotes political conservatism.
\emph{Personality and Social Psychology Bulletin}, \emph{38}(6),
808--820.

\hypertarget{ref-eskelson2011estimating}{}
Eskelson, B. N., Madsen, L., Hagar, J. C., \& Temesgen, H. (2011).
Estimating riparian understory vegetation cover with beta regression and
copula models. \emph{Forest Science}, \emph{57}(3), 212--221.

\hypertarget{ref-everitt2002cambridge}{}
Everitt, B., \& Skrondal, A. (2010). \emph{The cambridge dictionary of
statistics}. New York, NY: Cambridge University Press.

\hypertarget{ref-ferrari2004beta}{}
Ferrari, S., \& Cribari-Neto, F. (2004). Beta regression for modelling
rates and proportions. \emph{Journal of Applied Statistics},
\emph{31}(7), 799--815.

\hypertarget{ref-gallardo2012analysis}{}
Gallardo, A., Bovea, M. D., Colomer, F. J., \& Prades, M. (2012).
Analysis of collection systems for sorted household waste in Spain.
\emph{Waste Management}, \emph{32}(9), 1623--1633.

\hypertarget{ref-hayes2007using}{}
Hayes, A. F., \& Cai, L. (2007). Using heteroskedasticity-consistent
standard error estimators in OLS regression: An introduction and
software implementation. \emph{Behavior Research Methods}, \emph{39}(4),
709--722.

\hypertarget{ref-hiel2002explaining}{}
Hiel, A. V., \& Mervielde, I. (2002). Explaining conservative beliefs
and political preferences: A comparison of social dominance orientation
and authoritarianism. \emph{Journal of Applied Social Psychology},
\emph{32}(5), 965--976.

\hypertarget{ref-ho2015nature}{}
Ho, A. K., Sidanius, J., Kteily, N., Sheehy-Skeffington, J., Pratto, F.,
Henkel, K. E., \ldots{} Stewart, A. L. (2015). The nature of social
dominance orientation: Theorizing and measuring preferences for
intergroup inequality using the new SDO\(_7\) scale. \emph{Journal of
Personality and Social Psychology}, \emph{109}(6), 1003.

\hypertarget{ref-hubben2008societal}{}
Hubben, G. A. A., Bishai, D., Pechlivanoglou, P., Cattelan, A. M.,
Grisetti, R., Facchin, C., \ldots{} Tramarin, A. (2008). The societal
burden of HIV/AIDS in Northern Italy: An analysis of costs and quality
of life. \emph{AIDS Care}, \emph{20}(4), 449--455.

\hypertarget{ref-jost2003political}{}
Jost, J. T., Glaser, J., Kruglanski, A. W., \& Sulloway, F. J. (2003).
Political conservatism as motivated social cognition.
\emph{Psychological Bulletin}, \emph{129}(3), 339.

\hypertarget{ref-long2000using}{}
Long, J. S., \& Ervin, L. H. (2000). Using heteroscedasticity consistent
standard errors in the linear regression model. \emph{The American
Statistician}, \emph{54}(3), 217--224.

\hypertarget{ref-louis2003reflections}{}
Louis, W. R., Mavor, K. I., \& Terry, D. J. (2003). Reflections on the
statistical analysis of personality and norms in war, peace, and
prejudice: Are deviant minorities the problem? \emph{Analyses of Social
Issues and Public Policy}, \emph{3}(1), 189--198.

\hypertarget{ref-mcbee2010modeling}{}
McBee, M. (2010). Modeling outcomes with floor or ceiling effects: An
introduction to the tobit model. \emph{Gifted Child Quarterly},
\emph{54}(4), 314--320.

\hypertarget{ref-peplonska2012rotating}{}
Peplonska, B., Bukowska, A., Sobala, W., Reszka, E., Gromadzińska, J.,
Wasowicz, W., \ldots{} Ursin, G. (2012). Rotating night shift work and
mammographic density. \emph{Cancer, Epidemiology, Biomarkers, \&
Prevention}, \emph{21}(7), 1028--1037.

\hypertarget{ref-pratto2000social}{}
Pratto, F., Liu, J. H., Levin, S., Sidanius, J., Shih, M., Bachrach, H.,
\& Hegarty, P. (2000). Social dominance orientation and the
legitimization of inequality across cultures. \emph{Journal of
Cross-Cultural Psychology}, \emph{31}(3), 369--409.

\hypertarget{ref-pratto1994social}{}
Pratto, F., Sidanius, J., Stallworth, L. M., \& Malle, B. F. (1994).
Social dominance orientation: A personality variable predicting social
and political attitudes. \emph{Journal of Personality and Social
Psychology}, \emph{67}(4), 741.

\hypertarget{ref-pratto1998social}{}
Pratto, F., Stallworth, L. M., \& Conway-Lanz, S. (1998). Social
dominance orientation and the ideological legitimization of social
policy. \emph{Journal of Applied Social Psychology}, \emph{28}(20),
1853--1875.

\hypertarget{ref-pratto1997gender}{}
Pratto, F., Stallworth, L. M., \& Sidanius, J. (1997). The gender gap:
Differences in political attitudes and social dominance orientation.
\emph{British Journal of Social Psychology}, \emph{36}(1), 49--68.

\hypertarget{ref-rcore2017}{}
R Core Team. (2017). \emph{R: A language and environment for statistical
computing}. Vienna, Austria: R Foundation for Statistical Computing.
Retrieved from \url{https://www.R-project.org/}

\hypertarget{ref-rigby2005generalized}{}
Rigby, R. A., \& Stasinopoulos, D. M. (2005). Generalized additive
models for location, scale and shape,(with discussion). \emph{Applied
Statistics}, \emph{54.3}, 507--554.

\hypertarget{ref-rigby2017distributions}{}
Rigby, R., Stasinopoulos, M., Heller, G., \& De Bastiani, F. (2017).
\emph{Distributions for modelling location, scale, and shape: Using
GAMLSS in R. Unpublished draft}. Retrieved from
\url{gamlss.com/books-articles}

\hypertarget{ref-rosopa2013managing}{}
Rosopa, P. J., Schaffer, M. M., \& Schroeder, A. N. (2013). Managing
heteroscedasticity in general linear models. \emph{Psychological
Methods}, \emph{18}(3), 335.

\hypertarget{ref-smithson2013generalized}{}
Smithson, M., \& Merkle, E. C. (2013). \emph{Generalized linear models
for categorical and continuous limited dependent variables}. Boca Raton,
FL: CRC Press.

\hypertarget{ref-stasinopoulos2017flexible}{}
Stasinopoulos, M. D., Rigby, R. A., Heller, G. Z., Voudouris, V., \& De
Bastiani, F. (2017). \emph{Flexible regression and smoothing: Using
GAMLSS in R}. Boca Raton, FL: CRC Press.

\hypertarget{ref-venables2002modern}{}
Venables, W. N., \& Ripley, B. D. (2002). \emph{Modern applied
statistics with s}. New York, NY: Springer.

\hypertarget{ref-white2017freedom}{}
White II, M. H., \& Crandall, C. S. (2017). Freedom of racist speech:
Ego and expressive threats. \emph{Journal of Personality and Social
Psychology}, \emph{113}(3), 413.

\hypertarget{ref-wilson2013social}{}
Wilson, M. S., \& Sibley, C. G. (2013). Social dominance orientation and
right-wing authoritarianism: Additive and interactive effects on
political conservatism. \emph{Political Psychology}, \emph{34}(2),
277--284.

\hypertarget{ref-wooldridge2010econometric}{}
Wooldridge, J. M. (2010). \emph{Econometric analysis of cross section
and panel data}. Cambridge, MA: MIT press.






\end{document}
