\documentclass[english,man]{apa6}

\usepackage{amssymb,amsmath}
\usepackage{ifxetex,ifluatex}
\usepackage{fixltx2e} % provides \textsubscript
\ifnum 0\ifxetex 1\fi\ifluatex 1\fi=0 % if pdftex
  \usepackage[T1]{fontenc}
  \usepackage[utf8]{inputenc}
\else % if luatex or xelatex
  \ifxetex
    \usepackage{mathspec}
    \usepackage{xltxtra,xunicode}
  \else
    \usepackage{fontspec}
  \fi
  \defaultfontfeatures{Mapping=tex-text,Scale=MatchLowercase}
  \newcommand{\euro}{€}
\fi
% use upquote if available, for straight quotes in verbatim environments
\IfFileExists{upquote.sty}{\usepackage{upquote}}{}
% use microtype if available
\IfFileExists{microtype.sty}{\usepackage{microtype}}{}

% Table formatting
\usepackage{longtable, booktabs}
\usepackage{lscape}
% \usepackage[counterclockwise]{rotating}   % Landscape page setup for large tables
\usepackage{multirow}		% Table styling
\usepackage{tabularx}		% Control Column width
\usepackage[flushleft]{threeparttable}	% Allows for three part tables with a specified notes section
\usepackage{threeparttablex}            % Lets threeparttable work with longtable

% Create new environments so endfloat can handle them
% \newenvironment{ltable}
%   {\begin{landscape}\begin{center}\begin{threeparttable}}
%   {\end{threeparttable}\end{center}\end{landscape}}

\newenvironment{lltable}
  {\begin{landscape}\begin{center}\begin{ThreePartTable}}
  {\end{ThreePartTable}\end{center}\end{landscape}}

  \usepackage{ifthen} % Only add declarations when endfloat package is loaded
  \ifthenelse{\equal{\string man}{\string man}}{%
   \DeclareDelayedFloatFlavor{ThreePartTable}{table} % Make endfloat play with longtable
   % \DeclareDelayedFloatFlavor{ltable}{table} % Make endfloat play with lscape
   \DeclareDelayedFloatFlavor{lltable}{table} % Make endfloat play with lscape & longtable
  }{}%



% The following enables adjusting longtable caption width to table width
% Solution found at http://golatex.de/longtable-mit-caption-so-breit-wie-die-tabelle-t15767.html
\makeatletter
\newcommand\LastLTentrywidth{1em}
\newlength\longtablewidth
\setlength{\longtablewidth}{1in}
\newcommand\getlongtablewidth{%
 \begingroup
  \ifcsname LT@\roman{LT@tables}\endcsname
  \global\longtablewidth=0pt
  \renewcommand\LT@entry[2]{\global\advance\longtablewidth by ##2\relax\gdef\LastLTentrywidth{##2}}%
  \@nameuse{LT@\roman{LT@tables}}%
  \fi
\endgroup}


\ifxetex
  \usepackage[setpagesize=false, % page size defined by xetex
              unicode=false, % unicode breaks when used with xetex
              xetex]{hyperref}
\else
  \usepackage[unicode=true]{hyperref}
\fi
\hypersetup{breaklinks=true,
            pdfauthor={},
            pdftitle={Modeling Normative Aspects of Social Attitudes with Beta Regression},
            colorlinks=true,
            citecolor=blue,
            urlcolor=blue,
            linkcolor=black,
            pdfborder={0 0 0}}
\urlstyle{same}  % don't use monospace font for urls

\setlength{\parindent}{0pt}
%\setlength{\parskip}{0pt plus 0pt minus 0pt}

\setlength{\emergencystretch}{3em}  % prevent overfull lines

\ifxetex
  \usepackage{polyglossia}
  \setmainlanguage{}
\else
  \usepackage[english]{babel}
\fi

% Manuscript styling
\captionsetup{font=singlespacing,justification=justified}
\usepackage{csquotes}
\usepackage{upgreek}



\usepackage{tikz} % Variable definition to generate author note

% fix for \tightlist problem in pandoc 1.14
\providecommand{\tightlist}{%
  \setlength{\itemsep}{0pt}\setlength{\parskip}{0pt}}

% Essential manuscript parts
  \title{Modeling Normative Aspects of Social Attitudes with Beta Regression}

  \shorttitle{Beta Regression}


  \author{Mark H. White II\textsuperscript{1}}

  \def\affdep{{""}}%
  \def\affcity{{""}}%

  \affiliation{
    \vspace{0.5cm}
          \textsuperscript{1} University of Kansas  }

  \authornote{
    \newcounter{author}
    Author note will go here.

                      Correspondence concerning this article should be addressed to Mark H. White II. E-mail: \href{mailto:markhwhiteii@gmail.com}{\nolinkurl{markhwhiteii@gmail.com}}
                }


  \abstract{Abstract will go here.}
  \keywords{beta regression, hurdle models, gamlss, norms, social attitudes \\

    
  }





\usepackage{amsthm}
\newtheorem{theorem}{Theorem}
\newtheorem{lemma}{Lemma}
\theoremstyle{definition}
\newtheorem{definition}{Definition}
\newtheorem{corollary}{Corollary}
\newtheorem{proposition}{Proposition}
\theoremstyle{definition}
\newtheorem{example}{Example}
\theoremstyle{remark}
\newtheorem*{remark}{Remark}
\begin{document}

\maketitle

\setcounter{secnumdepth}{0}



\section{Statistical Background}\label{statistical-background}

\subsection{The Beta Distribution}\label{the-beta-distribution}

The beta distribution can be used to model the residuals in a
generalized linear model when the values of the dependent variable are
bounded \(0 < y < 1\) (Coxe, West, \& Aiken, 2013). The probability
density function (pdf) of the beta distribution is determined by two
parameters, \(\alpha\) and \(\beta\), that are called \enquote{shape}
parameters:

\begin{center}
$f(y;\alpha,\beta)={\Gamma(\alpha+\beta)\over\Gamma(\alpha)\Gamma(\beta)}y^{\alpha-1}(1-y)^{\beta-1}$
\end{center}

where \(\Gamma(.)\) is the gamma function. The two shape parameters pull
the mean toward zero and one, respectively. If \(\alpha\) is larger than
\(\beta\), the mean leans toward zero; if the reverse is true, the mean
leans toward one. One of the benefits of the beta distribution is that
it is flexible and can take a number of distributional shapes. ADD
FIGURE 1 HERE TO SHOW VARIOUS SHAPES.

The values of these parameters are not meaningful to applied
researchers, however. Rigby, Stasinopoulos, Heller, and De Bastiani
(2017) \enquote{reparameterized} the beta distribution to make it easier
understand in a regression framework (but see Ferrari \& Cribari-Neto,
2004 for a different parameterization). Instead of predicting \(\alpha\)
and \(\beta\), they parameterize the beta distribution with two
different parameters: \(\mu\) (called the \enquote{location} parameter)
and \(\sigma\) (called the \enquote{scale} parameter), where
\(\mu={\alpha\over(\alpha+\beta)}\) and
\(\sigma=\sqrt{1\over(\alpha+\beta+1)}\). \(\mu\) is equivalent to the
mean, and \(\sigma\) is related positively to the variance. (Note that
\(\sigma\) is \emph{not} the standard deviation, even though \(\sigma\)
refers to the standard deviation in many other contexts.) The variance
is equivalent to \(\sigma^2\mu(1-\mu)\). There are two important things
to note from this equation: First, the greater the \(\sigma\), the
greater the variance; second, the variance depends on the mean, which
means that beta regression, covered shortly, will be naturally
heteroskedastic.

Using this distribution in a regression framework cannot handle the
values zero and one, however. How can we model dependent variables that
also include observations at zero and one?

\subsection{Zero-One Inflated Beta
Distribution}\label{zero-one-inflated-beta-distribution}

Rigby et al. (2017) show that the beta distribution can be
\enquote{inflated} at zero or one---that is, zeros and ones can be
included in the distribution. When the dependent variable contains
zeroes and ones (i.e., \(0 \leq y \leq 1\)), the pdf for this beta
inflated distribution, \(\text{beinf}\), is:

\begin{center}
\[
\text{beinf}(y;\mu,\sigma,\nu,\tau) =
\begin{cases}
  p_0                             & \text{if } y = 0\\
  (1 - p_0 - p_1)f(y;\mu,\sigma)  & \text{if } 0 < y < 1\\
  p_1                             & \text{if } y = 1
\end{cases}
\]
\end{center}

for \(0 \leq y \leq 1\), where \(f(y;\mu,\sigma)\) is the beta pdf with
\(\mu\) and \(\sigma\) bounded \emph{between} zero and one. The two
additional parameters, \(\nu\) and \(\tau\), are mixture parameters.
\(p_0\) is the probability that a case equals zero, \(p_1\) is the
probability that a case equals one, and \(p_2\) (i.e.,
\(1 - p_0 - p_1\)) is the probability that the case comes from the beta
distribution. In terms of these two additional parameters,
\(p_0 = {\nu \over (1 + \nu + \tau)}\) and
\(p_1 = {\tau \over (1 + \nu + \tau)}\). Rearranging these
algebraically, \(\nu\) is the odds that a case is zero to being from the
beta distribution, \(\nu = p_0 / p_2\), and \(\tau\) is the odds that a
case is a one to being from the beta distribution, \(\tau = p_1 / p_2\).

\subsection{Beta Regression Models}\label{beta-regression-models}

The goal now is to predict these four parameters,
\(\mu, \sigma, \nu, \tau\), from any number of predictor variables. All
four parameters can be predicted by an identical set of predictors, none
of the same predictors, or anywhere in between. Both \(\mu\) and
\(\sigma\) have to be between zero and one, so we can use the logistic
link function to fit predicted values in this range; both \(\nu\) and
\(\tau\) have to be greater than zero, so we can use the log link
function to fit predicted values in this range. Imagine we have one
predictor, \(x\). We could include this variable as a predictor of all
four variables with the equations:

\begin{center}
$\text{log}({\mu \over 1 - \mu}) = \beta_{10} + \beta_{11}X$\\
$\text{log}({\sigma \over 1 - \sigma}) = \beta_{20} + \beta_{21}X$\\
$\text{log}(\nu) = \beta_{30} + \beta_{31}X$\\
$\text{log}(\tau) = \beta_{40} + \beta_{41}X$
\end{center}

Or, equivalently stated:

\begin{center}
$\mu={1\over1+e^{-(\beta_{10}+\beta_{11}X)}}$\\
$\sigma={1\over1+e^{-(\beta_{20}+\beta_{21}X)}}$\\
$\nu = e^{\beta_{30} + \beta_{31}X}$\\
$\tau = e^{\beta_{40} + \beta_{41}X}$
\end{center}

where \(e^x\) is the natural exponential function. A regression model
can also be used when the dependent variable contains zeros but no ones
(e.g., \(0 \leq y < 1\)) or when it contains ones but no zeros (e.g.,
\(0 < y \leq 1\)). Let \(c\) be the value---0 or 1---that is included.
The pdf is:

\begin{center}
\[
\text{beinf}_c(y;\mu,\sigma,\nu) =
\begin{cases}
  p_c                             & \text{if } y = c\\
  (1 - p_c)f(y;\mu,\sigma)        & \text{if } 0 < y < 1\\
\end{cases}
\]
\end{center}

where \(\nu = p_c / (1 - p_c)\) and everything else is the same as
above. The same link functions are used as above, but the fourth
parameter, \(\tau\), is not included, as the dependent variable can only
take on values of \(c\) or from the beta distribution. Lastly, if no
zeros or ones are observed, the beta distribution alone can be used as
the pdf. This results in a beta regression where the researcher is only
predicting the location, \(\mu\), and shape, \(\sigma\). Since there is
no mixture with values of being 0 or 1, the latter two mixture
parameters, \(\nu\) and \(\tau\), are not included.

\subsection{Modeling Bounded Variables Outside the Unit
Interval}\label{modeling-bounded-variables-outside-the-unit-interval}

These beta regression models are often used when dealing with rates and
proportions, given that these are naturally bounded \(0 \leq y \leq 1\)
(Buntaine, 2011; Eskelson, Madsen, Hagar, \& Temesgen, 2011; Gallardo,
Bovea, Colomer, \& Prades, 2012; Hubben et al., 2008; Peplonska et al.,
2012). The beta distribution is doubly bounded continuous distribution,
meaning that, although it is on a continuous scale, values cannot be
greater than the upper bound, \(u\), or lesser than the lower bound,
\(l\). In the case of the beta distribution, \(l = 0\) and \(u = 1\).

One can view many scales researchers use as doubly bounded and
continuous. Although researchers generally model variables on Likert
scales and sliding scales (e.g., feeling thermometers) as being
conditionally normally distributed, these variables are, by definition,
not strictly normal. Observations from a normal distribution can take on
any value on the real number line. Observations from a standard 7-point
Likert scale can only take on values 1 through 7. In studying
controversial issues like prejudice and politics, many participants
score at the lower or upper bounds, causing floor or ceiling effects; in
these cases, the assumptions of normality and homoskedasticity are
likely to be violated. One can explicitly take into account that the
response is doubly bounded and heteroskedastic by using the beta
regression models described above, using a simple linear rescaling of
the data.

If one observes a dependent variable limited between two bounds, there
is a straightforward way to rescale the variable to the
\(0 \leq y \leq 1\) range:

\begin{center}
$y'_i = (y_i - l) / (u - l)$
\end{center}

where \(y\) is the variable on the original scale, \(y'\) is the
rescaled variable, \(u\) is the upper bound (i.e., the largest possible
value on the scale), \(l\) is the lower bound (i.e., the smallest
possible value on the scale), and the \(i\) subscript denotes an
individual's score. On a standard 7-point Likert scale, \(l = 1\) and
\(u = 7\). This rescaling allows a researcher to explicitly model
conditional variance, floor effects, and ceiling effects using beta
regression.

\subsubsection{Conditional variance}\label{conditional-variance}

Having an outcome that is a doubly bounded continuous variable can
produce heteroskedasticity. One of the assumptions of an ordinary least
squares (OLS) regression is homoskedasticity---that the variance of the
errors are unrelated to any predictor or any linear combination of
predictors. A regression equation with one independent variable \(x\) is
often written as \(y_i = \beta_0 + \beta_1x_i + \epsilon_i\), where
\(\epsilon_i\) is how far one's osbserved value \(y_i\) is from their
predicted value (i.e., the residual). Let \(\hat{y_i}\) be the predicted
value for observation \(i\), where \(\hat{y_i} = \beta_0 + \beta_1x_i\).
We can simplify the equation to: \(y_i = \hat{y_i} + \epsilon_i\) where
the residuals \(\epsilon\) are normally distributed with a mean of zero
and some variance---that is, \(\epsilon \sim N(0, \sigma^2_\epsilon)\).
We can further simplify this to:
\(y_i|x_i \sim N(\hat{y_i}, \sigma^2_\epsilon)\), which means that each
observation of the dependent variable we make, given the predictors we
use, are normally distributed with a mean of that individual's predicted
score and some variance. The assumption of homoskedasticity is found in
that \(\sigma^2_\epsilon\) does \emph{not} have a subscript \(i\). This
means that there is one \emph{common variance} of the residuals. Imagine
one runs a regression and observes \(\hat{y_i} = 2.5 + 1.5 \times x_i\)
and \(\sigma^2_\epsilon = 9\). For the first individual in the data, say
we observe \(x_1 = 0\), while we observe \(x_2 = 5\) for the second
individual. The model then assumes that the first person comes from a
normal distribution with a mean of 4 (i.e., \(\hat{y_1}\)) and a
variance of 9 (i.e., \(\sigma^2_\epsilon\)), while the second person
comes from a normal distribution with a mean of 10 (i.e., \(\hat{y_2}\))
and that same variance of 9 (i.e., \(\sigma^2_\epsilon)\). The
assumption is violated when these variances are not constant across all
levels of the predictor variables (or any linear combination of the
predictor variables). The violation of this assumption can cause Type I
or Type II errors, depending on the type of heteroskedasticity and the
source {[}CITE{]}.

There are a number of ways to test for and address violations of this
assumption {[}CITE{]}. Many common ways of dealing with
heteroskedasticity treat it as a nuisance to be dealt with, such as
robust standard errors, a transformation of the dependent variable, or
weighted least squares {[}CITE{]}. Heteroskedasticity can come from
various sources {[}CITE{]}, one being that the variance is influenced by
one of the observed predictor variables. Instead of treating this as a
nuisance to be corrected for when calculating \(p\)-values, explicitly
modeling this conditional variance could be an interesting phenomenon
per se. Since the beta regression models predict a shape parameter,
\(\sigma\), a researcher can draw conclusions like: \enquote{As \(x\)
increases, the variance in \(y\) increases.} Since we generally focus on
the \emph{location} parameter in OLS regression, researchers are
essentially ignoring interesting, potentially theoretically-relevant
parts of their data.

\subsubsection{Floor and ceiling
effects}\label{floor-and-ceiling-effects}

Floor and ceiling effects occur when there are many observations at the
lower and upper bound of the scale, respectively (Everitt \& Skrondal,
2010). Variables displaying these effects are likely to violate numerous
assumptions of ordinary least squares regression, such as conditional
normality and homoskedasticity of variance. Variables displaying these
effects can be thought of as being censored or truncated; a common
regression technique for censored or truncated variables are Tobit
models (Smithson \& Merkle, 2013). Censoring occurs when all scores
beyond some threshold are recorded at that threshold. If a researcher is
measuring reaction times, censoring occurs if, for example, the
researcher decides that any times above five seconds are scored as five.
If many people score above five, this censoring can cause a ceiling
effect. Truncating occurs when people who score beyond some treshold are
excluded from the data. If the researcher simply does not record trials
that take longer than five seconds, then the data are considered to be
truncated. Tobit models can be useful in these situations. However,
standard Tobit models assume that the underlying dimension is still
normal. A good application of this is in measuring the abilities of
intelligent children (McBee, 2010). The test may be too easy for the
population, resulting in a ceiling effect. Nonetheless, the ability
being measured is assumed to be normally distributed.

In the realm of prejudice and politics, I argue that it is useful to
\emph{not} think of prejudice as coming from a latent normal
distribution. Instead, people engage in a decision-making process. Using
prejudice as an example, people decide whether or not they are going to
admit to feeling any prejudice (i.e., scoring at the floor or above the
floor); if they decide to admit any prejudice, then they decide how much
to report. As another example: When measuring attitudes toward a
polarizing political figure, people decide whether or not to respond
entirely in the negative (i.e., at the floor), entirely in the positive
(i.e., at the ceiling), or somewhere in between. This could result in a
bimodal distribution, with both ceiling and floor effects.

This decision making process can be modeled with the inflated beta
regression models described above (at zero and one, zero only, or one
only). While these are often referred to as \enquote{zero-one inflated
beta regression} models in papers on beta regression, a more general
name used for these types of models are \enquote{two-part} models (Coxe
et al., 2013); when one assumes a decision-making process behind the two
parts, they are often referred to as \enquote{hurdle} models in the
econometrics literature (Cameron \& Trivedi, 2005; Wooldridge, 2010).

Hurdle models are a generalization of the Type I Tobit model (Cragg,
1971). (But note that, unlike the standard Tobit model, the latent
dependent construct is assumed to be conditionally \emph{beta}
distributed, which can take on many distributional forms, Figure 1.)
They are often used to model count processes with excess zeros (see
Carlevaro, Croissant, \& Hoareau, 2017 for a review of applications).
The log likelihood function for these models can be separated in two
parts: First, a logistic regression modeling the probability of zero
versus greater than zero; second, a truncated negative binomial model
using just the cases with observed dependent values greater than zero.
The inflated beta regressions can be seen as hurdle models, as the
zero-and-one inflated model's log likelihood can also be separated in
two parts: First, a multinomial logistic regression modeling the
probability of observing a zero, one, or a value between zero and one;
Second, a beta regression model using just the cases with observed
values between zero and one. Similarly, The inflated beta regression at
\emph{either} zero \emph{or} one has a log likelihood function that can
be separated in two parts: First, a logistic regression modeling the
probability of observing where the inflation occurs or between zero and
one; Second, a beta regression modeling just the cases with dependent
values between zero and one (Rigby et al., 2017).

Since their likelihoods can be separated, this makes a critical
assumption: That the two processes are independent, given the observed
predictors. In other words, the error terms of each submodel are assumed
to be independent of one another. Dependencies between them can be
modeled (referred to as ``selection'' models; Carlevaro et al., 2017;
Wooldridge, 2010), but I will not demonstrate these models here, as they
are not extended to beta regression.

\section{Norms Produce Invariance and Ceiling and Floor
Effects}\label{norms-produce-invariance-and-ceiling-and-floor-effects}

Data from MTurk

\section{Implementation in GAMLSS}\label{implementation-in-gamlss}

Show example from FoRS data

\newpage

\section{References}\label{references}

\setlength{\parindent}{-0.5in} \setlength{\leftskip}{0.5in}

\hypertarget{refs}{}
\hypertarget{ref-buntaine2011does}{}
Buntaine, M. T. (2011). Does the Asian Development Bank respond to past
environmental performance when allocating environmentally risky
financing? \emph{World Development}, \emph{39}(3), 336--350.

\hypertarget{ref-cameron2005microeconometrics}{}
Cameron, A. C., \& Trivedi, P. K. (2005). \emph{Microeconometrics:
Methods and applications}. New York, NY: Cambridge university press.

\hypertarget{ref-carlevaro2016multiple}{}
Carlevaro, F., Croissant, Y., \& Hoareau, S. (2017). \emph{Multiple
hurdle Tobit models in R: The mhurdle package}. Retrieved from
\url{https://cran.r-project.org/web/packages/mhurdle/vignettes/mhurdle.pdf}

\hypertarget{ref-coxe2013generalized}{}
Coxe, S., West, S. G., \& Aiken, L. S. (2013). Generalized linear
models. In T. D. Little (Ed.), \emph{The Oxford handbook of quantitative
methods, volume 2} (pp. 26--51). New York, NY: Oxford University Press.

\hypertarget{ref-cragg1971some}{}
Cragg, J. G. (1971). Some statistical models for limited dependent
variables with application to the demand for durable goods.
\emph{Econometrica: Journal of the Econometric Society}, 829--844.

\hypertarget{ref-eskelson2011estimating}{}
Eskelson, B. N., Madsen, L., Hagar, J. C., \& Temesgen, H. (2011).
Estimating riparian understory vegetation cover with beta regression and
copula models. \emph{Forest Science}, \emph{57}(3), 212--221.

\hypertarget{ref-everitt2002cambridge}{}
Everitt, B., \& Skrondal, A. (2010). \emph{The cambridge dictionary of
statistics}. New York, NY: Cambridge University Press.

\hypertarget{ref-ferrari2004beta}{}
Ferrari, S., \& Cribari-Neto, F. (2004). Beta regression for modelling
rates and proportions. \emph{Journal of Applied Statistics},
\emph{31}(7), 799--815.

\hypertarget{ref-gallardo2012analysis}{}
Gallardo, A., Bovea, M. D., Colomer, F. J., \& Prades, M. (2012).
Analysis of collection systems for sorted household waste in Spain.
\emph{Waste Management}, \emph{32}(9), 1623--1633.

\hypertarget{ref-hubben2008societal}{}
Hubben, G. A. A., Bishai, D., Pechlivanoglou, P., Cattelan, A. M.,
Grisetti, R., Facchin, C., \ldots{} Tramarin, A. (2008). The societal
burden of HIV/AIDS in Northern Italy: An analysis of costs and quality
of life. \emph{AIDS Care}, \emph{20}(4), 449--455.

\hypertarget{ref-mcbee2010modeling}{}
McBee, M. (2010). Modeling outcomes with floor or ceiling effects: An
introduction to the tobit model. \emph{Gifted Child Quarterly},
\emph{54}(4), 314--320.

\hypertarget{ref-peplonska2012rotating}{}
Peplonska, B., Bukowska, A., Sobala, W., Reszka, E., Gromadzińska, J.,
Wasowicz, W., \ldots{} Ursin, G. (2012). Rotating night shift work and
mammographic density. \emph{Cancer, Epidemiology, Biomarkers, \&
Prevention}, \emph{21}(7), 1028--1037.

\hypertarget{ref-rigby2017distributions}{}
Rigby, R., Stasinopoulos, M., Heller, G., \& De Bastiani, F. (2017).
\emph{Distributions for modelling location, scale, and shape: Using
GAMLSS in R. Unpublished draft}. Retrieved from
\url{gamlss.com/books-articles}

\hypertarget{ref-smithson2013generalized}{}
Smithson, M., \& Merkle, E. C. (2013). \emph{Generalized linear models
for categorical and continuous limited dependent variables}. Boca Raton,
FL: CRC Press.

\hypertarget{ref-wooldridge2010econometric}{}
Wooldridge, J. M. (2010). \emph{Econometric analysis of cross section
and panel data}. Cambridge, MA: MIT press.






\end{document}
