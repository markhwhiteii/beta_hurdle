\documentclass[english,man]{apa6}

\usepackage{amssymb,amsmath}
\usepackage{ifxetex,ifluatex}
\usepackage{fixltx2e} % provides \textsubscript
\ifnum 0\ifxetex 1\fi\ifluatex 1\fi=0 % if pdftex
  \usepackage[T1]{fontenc}
  \usepackage[utf8]{inputenc}
\else % if luatex or xelatex
  \ifxetex
    \usepackage{mathspec}
    \usepackage{xltxtra,xunicode}
  \else
    \usepackage{fontspec}
  \fi
  \defaultfontfeatures{Mapping=tex-text,Scale=MatchLowercase}
  \newcommand{\euro}{€}
\fi
% use upquote if available, for straight quotes in verbatim environments
\IfFileExists{upquote.sty}{\usepackage{upquote}}{}
% use microtype if available
\IfFileExists{microtype.sty}{\usepackage{microtype}}{}

% Table formatting
\usepackage{longtable, booktabs}
\usepackage{lscape}
% \usepackage[counterclockwise]{rotating}   % Landscape page setup for large tables
\usepackage{multirow}		% Table styling
\usepackage{tabularx}		% Control Column width
\usepackage[flushleft]{threeparttable}	% Allows for three part tables with a specified notes section
\usepackage{threeparttablex}            % Lets threeparttable work with longtable

% Create new environments so endfloat can handle them
% \newenvironment{ltable}
%   {\begin{landscape}\begin{center}\begin{threeparttable}}
%   {\end{threeparttable}\end{center}\end{landscape}}

\newenvironment{lltable}
  {\begin{landscape}\begin{center}\begin{ThreePartTable}}
  {\end{ThreePartTable}\end{center}\end{landscape}}

  \usepackage{ifthen} % Only add declarations when endfloat package is loaded
  \ifthenelse{\equal{\string man}{\string man}}{%
   \DeclareDelayedFloatFlavor{ThreePartTable}{table} % Make endfloat play with longtable
   % \DeclareDelayedFloatFlavor{ltable}{table} % Make endfloat play with lscape
   \DeclareDelayedFloatFlavor{lltable}{table} % Make endfloat play with lscape & longtable
  }{}%



% The following enables adjusting longtable caption width to table width
% Solution found at http://golatex.de/longtable-mit-caption-so-breit-wie-die-tabelle-t15767.html
\makeatletter
\newcommand\LastLTentrywidth{1em}
\newlength\longtablewidth
\setlength{\longtablewidth}{1in}
\newcommand\getlongtablewidth{%
 \begingroup
  \ifcsname LT@\roman{LT@tables}\endcsname
  \global\longtablewidth=0pt
  \renewcommand\LT@entry[2]{\global\advance\longtablewidth by ##2\relax\gdef\LastLTentrywidth{##2}}%
  \@nameuse{LT@\roman{LT@tables}}%
  \fi
\endgroup}


\ifxetex
  \usepackage[setpagesize=false, % page size defined by xetex
              unicode=false, % unicode breaks when used with xetex
              xetex]{hyperref}
\else
  \usepackage[unicode=true]{hyperref}
\fi
\hypersetup{breaklinks=true,
            pdfauthor={},
            pdftitle={Modeling Social Attitudes with Beta Regression Hurdle Models},
            colorlinks=true,
            citecolor=blue,
            urlcolor=blue,
            linkcolor=black,
            pdfborder={0 0 0}}
\urlstyle{same}  % don't use monospace font for urls

\setlength{\parindent}{0pt}
%\setlength{\parskip}{0pt plus 0pt minus 0pt}

\setlength{\emergencystretch}{3em}  % prevent overfull lines

\ifxetex
  \usepackage{polyglossia}
  \setmainlanguage{}
\else
  \usepackage[english]{babel}
\fi

% Manuscript styling
\captionsetup{font=singlespacing,justification=justified}
\usepackage{csquotes}
\usepackage{upgreek}



\usepackage{tikz} % Variable definition to generate author note

% fix for \tightlist problem in pandoc 1.14
\providecommand{\tightlist}{%
  \setlength{\itemsep}{0pt}\setlength{\parskip}{0pt}}

% Essential manuscript parts
  \title{Modeling Social Attitudes with Beta Regression Hurdle Models}

  \shorttitle{Beta Hurdles}


  \author{Mark H. White II\textsuperscript{1}}

  \def\affdep{{""}}%
  \def\affcity{{""}}%

  \affiliation{
    \vspace{0.5cm}
          \textsuperscript{1} University of Kansas  }

  \authornote{
    \newcounter{author}
    Author note will go here.

                      Correspondence concerning this article should be addressed to Mark H. White II. E-mail: \href{mailto:markhwhiteii@gmail.com}{\nolinkurl{markhwhiteii@gmail.com}}
                }


  \abstract{Abstract will go here.}
  \keywords{beta regression, hurdle models, norms, social attitudes \\

    
  }





\usepackage{amsthm}
\newtheorem{theorem}{Theorem}
\newtheorem{lemma}{Lemma}
\theoremstyle{definition}
\newtheorem{definition}{Definition}
\newtheorem{corollary}{Corollary}
\newtheorem{proposition}{Proposition}
\theoremstyle{definition}
\newtheorem{example}{Example}
\theoremstyle{remark}
\newtheorem*{remark}{Remark}
\begin{document}

\maketitle

\setcounter{secnumdepth}{0}



\subsection{The Beta Distribution}\label{the-beta-distribution}

The beta distribution can be used to model the residuals in a
generalized linear model when the dependent values are bounded
\(0 < y < 1\). The probability density function (pdf) of the beta
distribution is determined by two parameters, \(\alpha\) and \(\beta\),
that are called \enquote{shape} parameters:

\begin{center}
$f(y;\alpha,\beta)={\Gamma(\alpha+\beta)\over\Gamma(\alpha)\Gamma(\beta)}y^{\alpha-1}(1-y)^{\beta-1}$
\end{center}

Where \(\Gamma(.)\) is the gamma function. The two shape parameters pull
the mean toward zero and one, respectively. These parameters are not
intuitive to applied researchers used to using regression and analysis
of variance.

Stasinopoulos and colleagues \enquote{reparameterized} the beta
distribution to make it easier understand in a regression framework.
Instead of predicting \(\alpha\) and \(\beta\), they parameterize the
beta distribution with two different parameters: \(\mu\) (called the
\enquote{location} parameter) and \(\sigma\) (called the \enquote{scale}
parameter), where \(\mu={\alpha\over(\alpha+\beta)}\) and
\(\sigma=\sqrt{1\over(\alpha+\beta+1)}\). \(\mu\) is equivalent to the
mean, and \(\sigma\) is related to the variance. \(\sigma\) is not the
standard deviation (even though \(\sigma\) refers to the standard
deviation in other contexts). The variance is equivalent to
\(\sigma^2\mu(1-\mu)\). There are two important things to note from this
equation: first, the greater \(\sigma\) is, the greater the variance is;
second, the variance depends on the mean. This means that beta
regression, covered shortly, will be naturally heteroskedastic.

But how can we model dependent variables that equal zero and/or one?

\subsection{Mixture Beta Distribution}\label{mixture-beta-distribution}

Stasinopoulos and colleagues show that the beta distribution can be
\enquote{inflated} at zero or one. However, I will not use the language
of \enquote{inflation,} as this distribution is more intuitively thought
of as a mixture distribution. When the dependent variable contains
zeroes and ones (i.e., \(0 \leq y \leq 1\)), the pdf for this beta
mixture, \(\text{bmix}\), is:

\begin{center}
\[
\text{bmix}(y;\mu,\sigma,\nu,\tau) =
\begin{cases}
  p_0                             & \text{if } y = 0\\
  (1 - p_0 - p_1)f(y;\mu,\sigma)  & \text{if } 0 < y < 1\\
  p_1                             & \text{if } y = 1
\end{cases}
\]
\end{center}

for \(0 \leq y \leq 1\), where \(f(y;\mu,\sigma)\) is the beta pdf with
\(\mu\) and \(\sigma\) bounded \emph{between} zero and one. The two
additional parameters, \(\nu\) and \(\tau\), are mixture parameters.
\(p_0\) is the probability that a case equals zero, \(p_1\) is the
probability that a case equals one, and \(p_2\) (i.e.,
\(1 - p_0 - p_1\)) is the probability that the case comes from the beta
distribution. In terms of these two additional parameters,
\(p_0 = {\nu \over (1 + \nu + \tau)}\) and
\(p_1 = {\tau \over (1 + \nu + \tau)}\). Rearranging these
algebraically, \(\nu\) is the odds that a case is zero compared to being
from the beta distribution, \(\nu = p_0 / p_2\), and \(\tau\) is the
odds that a case is a one compared to being from the beta distribution,
\(\tau = p_1 / p_2\).

\subsection{Beta Regression Hurdle
Models}\label{beta-regression-hurdle-models}

The goal now is to predict these four parameters,
\(\mu, \sigma, \nu, \tau\). Both \(\mu\) and \(\sigma\) have to be
between zero and one, so we can use the logistic link function to fit
predicted values in this range; both \(\nu\) and \(\tau\) have to be
greater than zero, so we can use the log link function to fit predicted
values in this range. Imagine we have one predictor, \(x\), for the
dependent variable \(y\). We could include this variable as a predictor
of all four variables with the equations:

\begin{center}
$\text{log}({\mu \over 1 - \mu}) = \beta_{10} + \beta_{11}X$\\
$\text{log}({\sigma \over 1 - \sigma}) = \beta_{20} + \beta_{21}X$\\
$\text{log}(\nu) = \beta_{30} + \beta_{31}X$\\
$\text{log}(\tau) = \beta_{40} + \beta_{41}X$
\end{center}

Or, equivalently stated:

\begin{center}
$\mu={1\over1+e^{-(\beta_{10}+\beta_{11}X)}}$\\
$\sigma={1\over1+e^{-(\beta_{20}+\beta_{21}X)}}$\\
$\nu = e^{\beta_{30} + \beta_{31}X}$\\
$\tau = e^{\beta_{40} + \beta_{41}X}$
\end{center}

Where \(e^x\) is the natural exponential function.

\newpage

\section{References}\label{references}

\setlength{\parindent}{-0.5in} \setlength{\leftskip}{0.5in}






\end{document}
